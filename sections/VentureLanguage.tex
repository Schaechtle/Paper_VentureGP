Venture is a compositional language for custom inference strategies that comes with a Scheme-like and a JavaScript-like front-end syntaxes.
Its implementation is based on on three concepts:
  (i) \emph{stochastic procedures} that specify and encapsulate random variables, analogously to conditional probability tables in a Bayesian network;
  (ii) \emph{execution traces} that represent (partial) execution histories and track the conditional dependencies of the random variables occurring therein; and
  (iii) \emph{scaffolds} that partition execution histories and factor global inference problems into sub-problems.
These building blocks provide a powerful and concise way to represent probability distributions, including distributions with a dynamically determined and unbounded set of random variables.
In this \paperOrChapter we will use only the four basic Venture directives: ASSUME, OBSERVE, SAMPLE and INFER.
\begin{itemize}
  \item ASSUME induces a hypothesis space for (probabilistic) models including random variables by binding the result of a supplied expression to a supplied symbol.
  \item Whereas in Scheme an expression is evaluated within an environment, in Venture an expression is evaluated within a (partial) trace of the model program.
    Thus, the value of an expression within a model program is a random variable, whose randomness comes from the distribution on possible execution traces of the program.
    The SAMPLE directive samples the value of the supplied expression within the current model program.
  \item OBSERVE constrains the supplied expression to have the supplied value.
    In other words, all samples taken after an OBSERVE are conditioned on the observed data.
  \item INFER uses the supplied inference program to mutate the execution trace.
    For a correct inference program, this will result approximate sampling from the true posterior on execution traces, conditioned on the model and constraints introduced by ASSUME and OBSERVE.
    The posterior on any random variable can then be approximately sampled by calling SAMPLE to extract values from the trace.
\end{itemize}

INFER is commonly done using the Metropolis--Hastings algorithm (MH)~\citep{metropolis1953equation}.
Many of the most popular MCMC algorithms can be interpreted as special cases of MH~\citep{andrieu2003introduction}.
We can outline the MH algorithm as follows.
The following two-step process is repeated as long as desired (say, for $T$ iterations):
First we sample $x^*$ from a proposal distribution $q$:
\begin{equation}
 x^* \sim q(x^* \mid x^{t});
\end{equation}
then we accept this proposal ($x^{t+1} \leftarrow x ^*$) with probability
\begin{equation}
  \alpha = \min \br{
    1,
    \frac{p(x^*) q(x^{t}\mid x^*)}{p(x^{t}) q(x^* \mid x^{t})}};
\end{equation}
if the proposal is not accepted then we take $x^{t+1} \gets x^t$.

Venture includes a built-in generic MH inference program which performs the above steps on any specified set of random variables in the model program.
In that inference program, partial execution traces play the role of $x$ above.




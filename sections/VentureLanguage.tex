Venture is a probabilistic programming language that allows both, the design of custom inference algorithms and arbitraly nested and hierarchical models. It comes with two different front end syntaxes, one Scheme-like and  one JavaScript-like.

Venture provides a powerful and concise way to represent probability distributions, including distributions with a dynamically determined and unbounded set of random variables.
In this \paperOrChapter we will use only the four basic Venture directives: \texttt{assume}, \texttt{observe}, \texttt{sample} and \texttt{infer}~\citep{mansinghka2014venture}.
\begin{itemize}
  \item \texttt{assume} induces a hypothesis space for (probabilistic) models including random variables by binding the result of a supplied expression to a supplied symbol.
  \item \texttt{sample} samples the value of the supplied expression within the current model program. Such an expression is evaluated within a trace of the model program.
    Thus, the value of an expression within a model program is a random variable, whose randomness comes from the distribution on possible execution traces of the program.
    The 
  \item \texttt{observe} constrains the supplied expression to have the supplied value.
    In other words, all samples taken after an \texttt{observe} are conditioned on the observed data.
  \item \texttt{infer} uses the supplied inference program changes the modeled probability distribution.
    This will result approximate sampling from the true posterior, conditioned on the model and constraints introduced by \texttt{assume} and \texttt{observe}.
    The posterior on any random variable can then be approximately sampled by calling \texttt{sample} to extract values from the modeled distribution.
\end{itemize}

\texttt{infer} is commonly done using the Metropolis--Hastings algorithm (MH)~\citep{metropolis1953equation}.
Many of the most popular MCMC algorithms can be interpreted as special cases of MH~\citep{andrieu2003introduction}.
We can outline the MH algorithm as follows.
The following two-step process is repeated as long as desired (say, for $T$ iterations):
First we sample $x^*$ from a proposal distribution $q$:
\begin{equation}
 x^* \sim q(x^* \mid x^{t});
\end{equation}
then we accept this proposal ($x^{t+1} \leftarrow x ^*$) with probability
\begin{equation}
  \alpha = \min \br{
    1,
    \frac{p(x^*) q(x^{t}\mid x^*)}{p(x^{t}) q(x^* \mid x^{t})}};
\end{equation}
if the proposal is not accepted then we take $x^{t+1} \gets x^t$.

Venture includes a built-in generic MH inference program which performs the above steps on any specified set of random variables in the model program.
In that inference program, partial execution traces play the role of $x$ above.




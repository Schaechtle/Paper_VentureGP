This paper has shown that it is feasible and useful to embed Gaussian processes
in higher-order probabilistic programming languages by treating them as a kind
of statistical memoizer. It has described classic GP regression with both fully
Bayesian and MAP inference in a hierarchical hyperprior, as well as state-of-the-art
applications to discovering symbolic structure in time series and to Bayesian optimization.
All the applications share a common 50-line Python GP library and require fewer than 20 lines
of probabilistic code each.

These results suggest several research directions. First, it will be important
to develop versions of {\tt gpmem} that are optimized for larger-scale
applications. Possible approaches include the standard low-rank approximations
to the kernel matrix that are popular in machine learning~\citep{bui2014tree} as well
as more sophisticated sampling algorithms for approximate conditioning of the
GP~\citep{lawrence2009efficient}.
Second, it seems fruitful to abstract the notion of a "generalizing" memoizer
from the specific choice of a Gaussian process model as the mechanism for
generalization. "Generalizing" or statistical memoizers with custom regression techniques could be broadly useful in performance engineering and scheduling systems.
The timing data from performance benchmarks could be run through a generalizing memoizer by default.
This memoizer could be queried (and its output error bars examined) to inform the best strategy
for performing the computation or predict the likely runtime of long-running jobs.
 Third, the structure learning application suggests follow-on research in information
retrieval for structured data. It should be possible to build a time series search engine
that can handle search predicates such as "has a rising trend starting around
1988" or "is perodic during the 1990s".
The variation on the Automated Statistician presented in this paper can provide ranked result
sets for these sorts of queries because it tracks posterior uncertainty over structure and also
because the space of structural patterns that it can handle is easy to modify by making small
changes to a short VentureScript program.


The field of Bayesian nonparametrics offers a principled, fully Bayesian
response to the empirical modeling philosophy in machine learning~\citep{ghahramani2013bayesian},
where Bayesian inference is used to encode a state of broad ignorance rather
than a bias stemming from strong prior knowledge. It is perhaps surprising that
two key objects from Bayesian nonparametrics, Dirichlet processes and Gaussian
processes, fit naturally in probabilistic programming as variants of
memoization~\citep{roy2008stochastic}. It is not yet clear if the same will be true
for other processes, e.g. Wishart processes, or hierarchical Beta processes. We hope that the results in this paper encourage the development of other nonparametric libraries for higher-order probabilistic programming languages.

\label{sec:special-case-gpmem}
We implement \gpmem\ by memoizing a target procedure in a wrapper that remembers previously computed values.
This comes with interesting implications:
from the standpoint of computation, a data set of the form $\{(x_i, y_i)\}$ can be thought of as a function $y = f_{\text{look-up}}(x)$, where $f_{\text{look-up}}$ is restricted to only allow evaluation at a specific set of inputs $x$.


Modelling the data set with a \ac{GP} then amounts to trying to learn a smooth function $f_\emu$ (``emu'' stands for ``emulator'') which extends $f$ to its full domain. We can then the incorporate observations in two different ways: we either are either told that the at a point \texttt{x} the value is \texttt{y}:
    \begin{lstlisting}
    observe f_emu ( x) = y
    \end{lstlisting}
Or we express this as
    \begin{lstlisting}
    predict f_compute ( x)
    \end{lstlisting}
The second expression has at least two benefits: (i) readability (in some cases), and (ii) amenability to active learning.
As to (i), the statistical code of creating a Gaussian process is replaced with a memoization-like idiom, which will be more familiar to programmers.
As to (ii), when using \gpmem, it is quite easy to decide incrementally which data point to sample next: for example, the loop from \texttt{x[1]} to \texttt{x[n]} could be replaced by a loop in which the next index \texttt{i} is chosen by a supplied decision rule.
In this way, we could use \gpmem\ to perform online learning using only a subset of the available data.

We illustrate the use of \gpmem\ in this context in a tutorial in Fig. \ref{fig:gpmem_tutorial}.

\begin{figure}
\centering
\begin{tabular}{ll}\small
% line 1
\begin{lstlisting}[mathescape,escapechar=\#]
assume (f_compute f_emu) =  gpmem( f)
sample (f_emu( array( -20, $\cdots$, 20)) 

             #\raisebox{-0.2\height}{\includegraphics[height=.5cm]{figs/oneline.png}}# $\sim \mathcal{N}(\mu,\mathbf{K})$
\end{lstlisting}
& \raisebox{-0.5\height}{\includegraphics[height=2.5cm]{figs/slide1_0pred.png}} \\ \hline
% line 2
\begin{lstlisting}[mathescape,escapechar=\#]
predict f_compute( 12.6)

sample (f_emu( array( -20, $\cdots$, 20)) 

\end{lstlisting}
 &  \raisebox{-0.5\height}{\includegraphics[height=2.5cm]{figs/slide1_1pred.png}}  \\ \hline
% line 3
 \begin{lstlisting}[mathescape,escapechar=\#]
predict f_compute( -6.4)

sample (f_emu( array( -20, $\cdots$, 20)) 
  
\end{lstlisting}
 &   \raisebox{-0.5\height}{\includegraphics[height=2.5cm]{figs/slide1_2pred.png}}
\end{tabular} 
%\put(-245,10){\line(1,0){200}}
\put(-40,98){\color{ForestGreen}\thicklines \vector(0,-1){20}}
\put(-117,11){\color{ForestGreen}\thicklines \vector(0,-1){20}}
\put(-59,-68){\thicklines \vector(0,-1){20}}
\put(-103,-43){\thicklines \vector(0,-1){20}}
\put(-91,-75){\thicklines \vector(0,1){20}}
\caption{\gpmem\ tutorial. The top shows a schematic of \gpmem. $f_{com} = $ \texttt{f\_compute} probes an outside resource. This can be expensive (top left). Every probe is memoized and improves the \ac{GP}-based emulator. Below the schematic we see a movie of the evolution of \gpmem's state of believe of the world given certain Venture directives.}
\label{fig:gpmem_tutorial}
\end{figure}

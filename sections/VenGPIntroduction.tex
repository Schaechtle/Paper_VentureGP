Probabilistic programming could be revolutionary for machine intelligence due to universal inference engines and the rapid prototyping for novel models~\citep{ghahramani2015probabilistic}. This propels the design and testing of new models as well as the incorporation of complex prior knowledge which currently is a difficult and time consuming task. Probabilistic programming languages aim to provide a formal language to specify probabilistic models in the style of computer programming and can represent any computable probability distribution as a program. In this work, we will introduce new features of Venture, a recently developed probabilistic programming language. We consider Venture the most compelling of the probabilistic programming languages because it is the first probabilistic programming language suitable for general purpose use~\citep{mansinghka2014venture}. Venture comes with scalable performance on hard problems and with a general purpose inference engine. The inference engine deploys Markov Chain Monte Carlo (MCMC) methods \citep[for an introduction, see][]{andrieu2003introduction}. MCMC lends itself to models with complex structures such as probabilistic programs or hierarchical Bayesian non-parametric models since they can provide a vehicle to express otherwise intractable integrals necessary for a fully Bayesian representation. MCMC is scalable, often distributable and also compositional. That is, one can arbitrarily chain MCMC kernels to infer over several hierarchically connected or nested models as they will emerge in probabilistic programming.

One powerful model rarely treated in probabilistic programming languages are Gaussian Processes (GPs). GPs are gaining increasing attention for representing unknown functions by posterior probability distributions in various fields such as machine learning, signal processing, computer vision and bio-medical data analysis. Making GPs available in probabilistic programming is crucial to allow a language to solve a wide range of problems. Hard problems include but are not limited to hierarchical prior construction~\citep{neal1997monte}, Bayesian Optimization~\citep{snoek2012practical} and systems for inductive learning of symbolic expressions such as the one introduced in the Automated Statistician project~\citep{duvenaud2013structure,lloyd2014automatic}. Learning such symbolic expressions is a hard problem that requires careful design of approximation techniques since standard inference method do not apply.

In the following, we will present GP memoization (\gpmem\ ) as a novel probabilistic programming technique that solves such hard problems. \gpmem\ introduces a statistical alternative to standard memoization.  Our contribution is threefold: 
\begin{itemize}
\item we introduce an efficient implementation of \gpmem\ in form of a self-caching wrapper  that remembers previously computed values;
\item we illustrate the statistical emulator that \gpmem\ produces and how it improves with every data-point that becomes available; and
 \item  we show how one can solve hard problems  of state-of-the-art machine learning related to GP  using \gpmem\ in a Bayesian fashion and with only a few lines of Venture code.
\end{itemize}

We evaluate the contribution on problems posed by the GP community using real world and synthetic data by assessing quality in terms of posterior distributions of symbolic outcome and in terms of the residuals produced by our probabilistic programs. 
The \paperOrChapter\ is structured as follows, we will first provide some background on memoization. We will explain programming in Venture and provide a brief introduction to GPs. We introduce \gpmem\ and its use in probabilistic programming and Bayesian modeling. Finally, we will show how we can apply \gpmem\ on problems of causally structured hierarchical priors for hyper-parameter inference, structure discovery for Gaussian Processes and Bayesian Optimization including experiments with real world and synthetic data.


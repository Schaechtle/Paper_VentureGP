In a Bayesian treatment of  hyper-parameter learning for \ac{GP}s,
we can write the probability of the hyper-parameters of a GP as
defined above, given covariance function $\mathbf{K}$ as:
\begin{equation}
\label{eq:hyperProbability}
P(\bm{\theta} \mid \mathbf{D,K}) = \frac{P(\mathbf{D} \mid \bm{\theta}, \mathbf{K})P(\bm{\theta} \mid  \mathbf{K})}{P(\mathbf{D} \mid \mathbf{K})}.
\end{equation}
Let the $\mathbf{K}$ be the sum of a smoothing and a white noise (WN) kernel. For this case, Neal~\citeyearpar{neal1997monte} suggested the problem of outliers in data as a use-case for a hierarchical Bayesian treatment of Gaussian processes\footnote{In Neal's work \citeyearpar{neal1997monte} the sum of an SE plus a constant kernel is used. We keep the WN kernel for illustrative purposes.}. The work suggests a hierarchical system of hyper-parameterization. Here, we draw hyper-parameters from a $\Gamma$ distributions:
\begin{equation}
\label{eq:hyper-ell}
\ell^{(t)} \sim \Gamma(\alpha_1,\beta_1),\;\sigma^{(t)} \sim \Gamma(\alpha_2,\beta_2)
\end{equation} 
and in turn sample the $\alpha$ and $\beta$ from $\Gamma$ distributions as well:
\begin{equation}
\label{eq:hyper-alpha}
\alpha_1^{(t)} \sim \Gamma(\alpha^1_{\alpha},\beta^1_{ \alpha } ),\; \alpha_2^{(t)} \sim \Gamma(\alpha^2_{\alpha},\beta^2_{\alpha}),\cdots
\end{equation}
% the input below will be empty, if we're in the paper. For my thesis, it will contain a causal network of the hyper-parameters.
One can represent this kind of model using \gpmem\ (Fig. \ref{fig:neal_tutorial}).
Neal provides a custom inference algorithm setting and evaluates it using the following synthetic data problem. Let $f$ be the underlying function that generates the data:
\begin{equation}
f(x) =  0.3 + 0.4 x + 0.5 \sin(2.7x) + \frac{1.1}{(1+ x^2)} + \eta \;\;\; with\;\;\eta \sim \mathcal{N}(0,\sigma)
\end{equation}
We synthetically generate outliers by setting $\sigma = 0.1$ in $95\%$ of the cases and to $\sigma = 1$ in the remaining cases. \gpmem\  can capture the true underlying function within only 100 MH steps on the hyper-parameters to get a good approximation for their posterior. Note that Neal devises an additional noise model and performs a large number of Hybrid-Monte Carlo and Gibbs steps.  



\begin{figure}
\renewcommand{\arraystretch}{0.1}% Tighter
\centering \footnotesize
\begin{tikzpicture}
\node[] (start) {};
\node[left=1cm of start] (sfsigma) {\includegraphics[height=2cm]{figs/hypers_sf_sigma.png}};
\node[right=1cm of start] (sfell) {\includegraphics[height=2cm]{figs/hypers_sf_ell.png}};
\node[above=1cm of start] (hyper) {(a) $P(\bm{\theta} = \{\ell,sf,\sigma\} \mid
\mathbf{D},\Krv)$};
\node[left=0.cm of sfsigma] (ell) {$\ell$};
\node[below=0.cm of sfsigma] (sf1) {$sf$};
\node[left=0.cm of sfell] (sigma) {$\sigma$};
\node[below=0.cm of sfell] (sf2) {$sf$};






%%%%%%%%%%%%%%%%%%%%%%%%%%%%%%%%%%%%%%%%%%%%%%%%%%%%%%%%%%%%
%%%%%%%%%%%%  Code             %%%%%%%%%%%%%%%%%%%%%%%%%%%%%
%%%%%%%%%%%%%%%%%%%%%%%%%%%%%%%%%%%%%%%%%%%%%%%%%%%%%%%%%%%%

\node[below=1.0cm of start] (b_code){
\footnotesize\begin{lstlisting}[mathescape,escapechar=\@]
// Define data and look-up function
define data      = array(array(-1.87, 0.13),@\ldots@, array(1.67,0.81));
assume f_look_up = proc(index) {lookup(data, index)};
\end{lstlisting}
};
\node[below=-0.65cm of b_code,xshift=-0.58cm] (c_code){
\footnotesize\begin{lstlisting}[mathescape,escapechar=\@
]
// Initialize hyper-priors
assume alpha_sf @$\sim$@ gamma(5,1)                 #"hyperhyper";
assume beta_sf  @$\sim$@ gamma(5,1)                 #"hyperhyper";
assume alpha_l  @$\sim$@ gamma(5,1)                 #"hyperhyper";
assume beta_l   @$\sim$@ gamma(5,1)                 #"hyperhyper";@\vspace{1mm}@
assume sf       @$\sim$@ gamma(alpha_sf, beta_sf))) #"hyper";
assume l        @$\sim$@ gamma(alpha_l, beta_l)))   #"hyper";
assume sigma    @$\sim$@ gamma(5,1))                #"hyper"; 
\end{lstlisting}
};
% (d) & (e)
\node[below=-0.65cm of c_code,xshift=-0.65cm] (d_code){
\footnotesize\begin{lstlisting}[mathescape,escapechar=\@]
// Initialize covariance function
assume se = @\se@;
assume wn = @\wn@;
assume composite_covariance = gp_cov_sum(se, wn);
// Create a prober and emulator using gpmem
assume (f_compute, f_emu) =
    gpmem(f_look_up, composite_covariance);@\vspace{1mm}@
sample f_emu(array(-2, $\cdots$, 2));
\end{lstlisting}
};
% (f)
\node[below=0.16cm of d_code,xshift=-0.18cm] (e_code){
\footnotesize\begin{lstlisting}[mathescape,escapechar=\@]

// Observe all data points
for (x,y) in data {
                   observe f_emu(x) = y};
// Or: probe all data points
for (x,_) in data {
                   predict f_compute(x)};@\vspace{1mm}@
sample f_emu(array(-2, $\cdots$, 2));
\end{lstlisting}
};
% (g)
\node[below=-0.0cm of e_code,xshift=-0.77cm] (f_code){
\footnotesize\begin{lstlisting}[mathescape,escapechar=\@]
// Metropolis-Hastings
infer repeat(100, do(
                mh(#"hyperhyper", steps=1),
                mh(#"hyper",      steps=3)));@\vspace{1mm}@
sample f_emu(array(-2, $\cdots$, 2));
\end{lstlisting}
};

\node[below=0.5cm of f_code,xshift=0.3cm] (g_code){
\footnotesize\begin{lstlisting}[mathescape,escapechar=\@]
// Optimization
infer gradient-ascent(#"hyper", steps=10);

sample f_emu(array(-2, $\cdots$, 2));
\end{lstlisting}
};



% samples/curve images
\node[below=1.1cm of c_code,xshift=6cm] (e_pic){
\includegraphics[height=3.4cm]{figs/neal_example_before_observation.jpg}
};
\node[below=-0.4cm of e_pic] (f_pic){
\includegraphics[height=3.4cm]{figs/neal_example_after_observation.jpg}
};
\node[below=-0.4cm of f_pic] (g_pic){
\includegraphics[height=3.4cm]{figs/neal_Bayesian.jpg}
};
\node[below=-0.4cm of g_pic] (h_pic){
\includegraphics[height=3.4cm]{figs/neal_example_map_inference_alpha0p01_iter15.jpg}
};

% horizontal lines
% (a)
\draw 
  ([xshift=-0.2cm,yshift=-0.6cm]b_code.north west) --
([xshift=0.8cm,yshift=-0.6cm]b_code.north east); 
% (b)
\draw 
  ([xshift=-0.2cm,yshift=0.08cm]b_code.south west) -- ([xshift=0.8cm,yshift=0.08cm]b_code.south east);   
% (c)
 \draw 
  ([xshift=-0.2cm,yshift=-3.17cm]b_code.south west) --
([xshift=0.8cm,yshift=-3.17cm]b_code.south east); 
% (d)
 \draw 
  ([xshift=-0.2cm,yshift=-4.78cm]b_code.south west) --
([xshift=0.8cm,yshift=-4.78cm]b_code.south east); 
% (e)
 \draw 
  ([xshift=-0.2cm,yshift=-7.7cm]b_code.south west) -- ([xshift=0.8cm,yshift=-7.7cm]b_code.south east); 
% (f)
 \draw 
  ([xshift=-0.2cm,yshift=-10.8cm]b_code.south west) -- ([xshift=0.8cm,yshift=-10.8cm]b_code.south east); 
% (g)
 \draw 
  ([xshift=-0.2cm,yshift=-14.1cm]b_code.south west) -- ([xshift=0.8cm,yshift=-14.1cm]b_code.south east); 

% labels
\node[below=1.9cm of start,xshift=-6.5cm] (b) {(b)};
\node[below=1.4cm of b] (c) {(c)};
\node[below=1.9cm of c] (d) {(d)};
\node[below=1.7cm of d] (e) {(e)};
\node[below=2.6cm of e] (f) {(f)};
\node[below=2.9cm of f] (g) {(g)};
\node[below=2.6cm of g] (h) {(h)};
\end{tikzpicture}


\caption{Regression with outliers and hierarchical prior structure.}
\label{fig:neal_tutorial}
\end{figure}




A \emph{probabilistic programming language} is a language for defining
probabilistic generative models and inference on those models.  For us, a
\emph{model program} means an executable description of a generative model.  An
\emph{inference program} is an executable description of an inference algorithm,
which may also make use of randomness.  In this paper, our examples will use the
Venture probabilistic programming language~\citep{mansinghka2014venture}, but the concepts
illustrated apply to probabilistic programming in general.  Notably, in Venture,
the sample space for all random variables occurring in a model is the set of all
possible execution traces of the model program (see \citep[\S4]{mansinghka2014venture} and
Section \ref{sec:interactivity}).  In fact, we take this as a definition: for a
probabilistic program written in Venture, we define the \emph{model program} to
consist of all instructions (synonymously, \emph{directives}) which enter the
execution trace, and the \emph{inference program} to consist of all directives,
including untraced directives which manipulate the existing trace.

The range of models describable as probabilistic programs is very broad,
containing graphical models (including plate notation) as a special case.
There is a precise sense in which any computable generative model can be
expressed as a probabilistic program~\citep{ackerman2010computability}.  The power of the
modelling language comes not from executability of models---in other words,
not from the ability to perform \emph{forward-sampling}---but from the rich
additional layers of description which allow automatic reasoning (such as
automatic dependency tracking) and semi-automatic reasoning (such as
exploratory inference programming) about models.

The range of inference algorithms describable as probabilistic programs is also
very broad.  General-purpose versions of rejection sampling, Gibbs sampling,
Metropolis--Hastings, mean-field, and slice sampling are built into Venture as
primitives~\citep{mansinghka2014venture}.  Custom inference algorithms can be written as
programs that manipulate traces: below we implement Thompson sampling as a
custom inference program.

Listing \ref{lst:pp-first} shows a simple probabilistic program containing only
modelling instructions:

\lstinputlisting[frame=single,label=lst:pp-first,caption={
  A simple probabilistic program (written in Venture) containing only modelling
  instructions.
}]{code/pp_first_example_v2.scm}
This program reads as follows: It is raining (R) with probability $0.3$.
If it is raining, then the sprinkler (S) is never on; if it is not raining,
then the sprinkler is on with probability $0.8$.  The probability that the
grass is wet (G) depends on both whether it is raining and whether the
sprinkler is on, according to the table
\begin{center}
  \begin{tabular}{|c|c|c|}
    \multicolumn{1}{c}{} & \multicolumn{1}{c}{R} & \multicolumn{1}{c}{$\neg\text{R}$} \\\cline{2-3}
    \multicolumn{1}{c|}{S} & 0.99 & 0.6 \\\cline{2-3}
    \multicolumn{1}{c|}{$\neg\text{S}$} & 0.95 & 0.1 \\\cline{2-3}
  \end{tabular}
\end{center}
Finally, the humidity is a continuous random variable whose parameters depend on
whether it is raining: $\Ncal(95, 2)$ (normal with mean $95$ and variance $2$)
if it is raining, and $\Ncal(50, 20)$ if it is not raining.

The program in Listing \ref{lst:pp-first} induces the following directed
graphical model:
\begin{center}
\begin{asy}
size(0,5cm);
pair R = (-3,3),
     S = (3,3),
     G = (3,-3),
     H = (-3,-3);
real radlength = 0.9;
guide circ = circle((0,0), radlength);

label("R", R);
label("S", S);
label("G", G);
label("H", H);

draw(shift(R) * circ);
draw(shift(S) * circ);
draw(shift(G) * circ);
draw(shift(H) * circ);

pair rad(real theta) { return radlength * exp(I*theta*pi/180); }
draw((R+rad(0))--(S+rad(180)), Arrow);
draw((R+rad(-45))--(G+rad(135)), Arrow);
draw((S+rad(-90))--(G+rad(90)), Arrow);
draw((R+rad(-90))--(H+rad(90)), Arrow);

label("Categorical CPD", R+rad(180), W);
label("Categorical CPD", S+rad(0), E);
label("Gaussian CPD", H+rad(180), W);
label("Categorical CPD", G+rad(0), E);

shipout(bbox(0.5cm));
\end{asy}
\end{center}

Not all model programs can be described by a graphical model: due to
branching and recursion, the set of random variables may vary from
execution to execution, as may the dependency structure of even a fixed set
of variables.  In fact, the class of computable distributions that can be
described by a graphical model is precisely the class of models that can be
written as a probabilistic program with fixed control flow.  Note that the
program in Listing \ref{lst:pp-first}, while it does not have fixed control
flow as written, has a finite number of possible control paths and thus can
be rewritten with fixed control flow by table lookup into an enumeration of
all possible paths.  There are other cases in which the class of models
expressible in an existing framework can be seen as those arising from a
restricted class of probabilistic programs; one such case is plate notation
in graphical models, which corresponds to probabilistic programs with fixed
control flow which employ memoization via \texttt{mem} (see
\citep[\S2.1]{goodman2008church}) and treat the value of the memoized procedure at each
input as a random variable.  However, the full class of models expressible as
probabilistic programs does not correspond to an existing notational framework.
This is not surprising: we would expect that a computationally universal class
of models requires a computational language to describe.\footnote{
  The language of probabilistic Turing machines is also ``universal'' in
  the sense that it could describe a program for forward-sampling at the
  machine level.  But such a description lacks many of the useful
  characteristics of a probabilistic program: semantics for what the random
  variables are and what they depend on, and abstraction and modularity to
  allow for local changes to the model program.
}

Suppose the value of a random variable is observed---say, $\text{H} = 85$---and
we wish to infer values of other random variables.  Rejection sampling of full
executions of the model program, which in this context we call \emph{global
rejection sampling}, produces joint samples from the posterior distribution
$P\pn{\text{R}, \text{S}, \text{G} \mvert \text{H}=85}$; it is achieved in
Venture by the directive
\begin{lstlisting}
(infer (rejection default all 1))
\end{lstlisting}
Global rejection sampling is essentially the only inference program that
can be employed on a black-box model program with no accompanying source
code or inspection infrastructure.  Our model, however, is not a black box;
due to the richness of the description language, a wide range of inference
programs are readily implementable.  Rejection sampling can be performed on
one random variable at a time, conditioned on the current values of the
other variables, resulting in a Markov chain Monte Carlo (MCMC) inference
routine whose state is a random tuple $(\text{R}, \text{S}, \text{G})$
whose distribution, after sufficiently many transitions, is approximately
the posterior.
%Inference on a single variable, randomly chosen from the
%collection of random variables in the model, is achieved in Venture by the
%directive
%\begin{lstlisting}
%infer (rejection default one 1)
%\end{lstlisting}
Note that inference on one variable at a time allows the programmer to
choose the order in which variables are inferred, or to choose not to infer
the values of variables on which the variable of interest does not depend.
This need not be done manually: Venture dynamically tracks dependencies between
random variables, which, for example, allows one to run Metropolis--Hastings
inference on a single random variable $X$ in which proposals only modify the
values of those variables that depend on $X$:
\begin{lstlisting}
(infer (mh default one 1))
\end{lstlisting}
(Here the keywords \texttt{default one} mean that $X$ is chosen randomly from
the set of random variables in the model, but it is also possible to specify an
explicit choice of $X$ by supplying different arguments to $\texttt{mh}$.)
While standard inference routines such as rejection sampling and
Metropolis--Hastings are built into Venture, inference in general can be (and in
Venture, is) fully programmable, allowing for composite inference strategies
which use different inference subroutines on different parts of the model at
different times, as well as dynamic choices of inference strategy which depend
on state accrued during the execution of the model program and inference
thereupon.

%\begin{verbatim}
%* Probabilistic programming in general, briefly (note: this is maximally
%vague, so this outline is not complete until I have more specificity on
%this item)
%  + vkm: 1-2 pages; just use a simple program as an example, and explain
%  that tutorial-style.
%  + "this probprog induces this directed graphical model; equiv to fixed
%  control flow"
%\end{verbatim}
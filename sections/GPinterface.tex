% 
% Who is a candidate reader?
%   I would say that there are two candidate readers. 
%     1. an ML person with background in nonparametrics. This reader does not care
%     about implementation (but she should!). We need to convince her that SPs are
%     a good way to think of GPs 
%     2. people with prob prog background that before reading our paper,
%     did not care about GPs
% 
% What is the idea that they are supposed to gain?
%   The idea that they are supposed to gain is that SPs as the basic units of
%   computation in Venture are good way to talk and think about GPs. 
% 
% There were too many irrelevant concepts at once.
%   I tried to divide what was there into relevant and irrelevant.
%   Relevant:
%     SP
%     GPSP
%     GPout(P)SP
%     MakeOutput(P)SP
%   Irrelevant:
%     AAA
%     PSP 
%     GPSPAux/Aux
% 
% Decision: I removed the PSP here - resulting in the description being somewhat
% unfaithful the implementation because GPoutPSP will be described as GPoutSP.
% PSP being  a kind of SP, I think that we are still telling a version of
% the truth. 
% 
% Issues:
%   - first paragraph is too assertive
%   - transition from first to second paragraph is not smooth
% 
Random samples from a \ac{GP}, like all random samples in Venture programs,
are generated with the invocation of \ac{SP}s. \ac{SP}s are the basic units of
computation in Venture. They provide a general framework to think about
probabilities that is particularly handy for \ac{GP}s. This framework does not
only define the implementation of a \ac{GP} model and possible inference over it but also
the representation and formulation of a \ac{GP} as a probabilistic model.
This aspect is somewhat different from the view of traditional machine learning,
seeing the software engineering and programming aspects only as a vehicle to
realize and communicate an idea.  In contrast, the language of samplers and \ac{SP}s provide a
different view point to what a \ac{GP} can and cannot compute.

\ac{SP}s in general accept input arguments that are values in Venture and sample output
values given those inputs. They can be deterministic or random. If an \ac{SP} is random
then it can simulate from a family of distributions. In addition to simulating,
a random \ac{SP} may be able to report the log-density of an output given an input.
Different \ac{SP}s are used to construct the \ac{GP} interface:
\begin{enumerate}
  \item MakeGPOutputSP is a deterministic \ac{SP} whose output is a
    \ac{GP}-sampler, namely the GPSP, given a covariance and a mean function as input. 
\item GPSP of is type SP\@. GPSP is responsible for tracking the sufficient statistics
from the applications of GPSP and for evaluating the log density of all those
applications as a block.
\item GPOutputPSP of type Random PSP\@.  It is this SP that actually samples
regression output at given input.
\end{enumerate}

Random samples from a \ac{GP}, like all random samples in Venture programs,
are generated with the invocation of \ac{SP}s.
Such \ac{SP}s accept input arguments that are values in Venture and sample output
values given those inputs. A \ac{PSP} is the most basic unit of computation in
Venture. It can be deterministic or random. If a \ac{PSP} is random then it can simulate from a family of
distributions. In addition to simulating,
a random PSPs may be able to report the log-density of an output given an input.
Different \ac{PSP}s and \ac{SP}s are used to construct the \ac{GP} interface:
\begin{enumerate}
\item MakeGPOutputPSP is a deterministic PSP whose output is a \ac{GP}, namely the
GPSP, given a
covariance and a mean function as input. Inputting a mean function is optional,
if it is not supplied, it is set to constant 0.

\item GPSP of is type SP. GPSP can \ac{AAA}. \ac{AAA}-\ac{SP}s can keep track of
their outputs and report
the joint log-density of all of its applications. This is crucial so that we do
not need to visit all of the dependent units of computation to explicitly account for their
log-densities. For Venture, the construction of the scaffold, that is the
factoring of global inference problems into coherent sub-problems, can stop upon
reaching an \ac{AAA} node. GPSP is responsible for tracking the sufficient statistics
from the applications of GPSP and for evaluating the log density of all those
applications as a block. It also constructs the auxiliary SP (SPAux).
\item GPSPAux of type SPAux. An auxiliary SP is stateless but comes with an auxiliary
store that carries mutable information. In case of the \ac{GP} it tracks the
samples that were incorporated.
\item GPOutputPSP of type Random PSP.  It is this SP that actually samples
regression output at given input.
\end{enumerate}


%procedures that return other stochastic procedures and implement this optimization are said to be
%absorbing at applications, often abbreviated AAA.




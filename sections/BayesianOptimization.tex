Bayesian optimization casts the problem of finding the global maximum of an unknown function as a hierarchical decision problem~\citep{ghahramani2015probabilistic}.
Evaluating the actual function may be very expensive, either in computation time or in some other resource.
For one example, when searching for the best configuration for the learning algorithm of a large convolutional neural network, a large amount of computational work is required to evaluate a candidate configuration, and the space of possible configurations is high-dimensional.
Another common example, alluded to in Section \ref{sec:gpmem-broader}, is data acquisition: for machine learning problems in which a large body of data is available, it is often desirable to choose the right queries to produce a data set on which learning will be most effective.
In continuous settings, many Bayesian optimization methods employ GPs~\citep[e.g.][]{snoek2012practical}.

To demonstrate Bayesian optimization with \gpmem\, we have implemented a version of Thompson sampling using GPs in Venture.
Thompson sampling~\citep{thompson1933likelihood} is a widely-used Bayesian framework for solving exploration-exploitation problems.
Our implementation has two notable features: (i) the ability to search over a broader space of contexts than the parametric families that are typically used, and (ii) the parsimony of the resulting probabilistic program.


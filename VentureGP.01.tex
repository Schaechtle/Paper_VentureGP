\documentclass[twoside,11pt]{article}

% actual style
\usepackage{jmlr2e}

% utils
\usepackage[colorlinks=true,citecolor=MidnightBlue,urlcolor=blue]{hyperref} % remove for Arxiv
\usepackage{url}
\usepackage{natbib}
\usepackage[nolist]{acronym}
\usepackage{caption}
\usepackage{subcaption}

% plotting
\usepackage[svgnames,dvipsnames]{xcolor} % Specify colors by their 'svgnames'
\usepackage{tikz}
\usetikzlibrary{bayesnet}
\usepackage{pgfplots}
\usepackage[abs]{overpic}

% pseudo-code
\usepackage{algorithm}
\usepackage{algpseudocode}

% Venture code boxes
\usepackage{listings}
\usepackage[framemethod=TikZ]{mdframed}% http://ctan.org/pkg/mdframed
\usepackage{setspace}
\usepackage{inconsolata} % will fit long lines into the box:w

\newdimen\linenumbersep
\newcommand{\linenumber}[1]{%
  \linenumbersep 4pt%
  \advance\linenumbersep\mdflength{innerleftmargin}%
  \advance\linenumbersep\mdflength{innerlinewidth}%
  \advance\linenumbersep\mdflength{middlelinewidth}%
  \advance\linenumbersep\mdflength{outerlinewidth}%
  \advance\linenumbersep\mdflength{linewidth}%
  \makebox[0pt][r]{{\rmfamily\tiny#1}\hspace*{\linenumbersep}}}



% Venture syntax highlighting for code boxes
\lstdefinelanguage{Venture}{
    alsoletter={\&,=,!,?}
    keywordstyle=\color{black},
    morecomment=[l]{//},
    commentstyle=\color{mygreen},
    keywordstyle=[2]\color{DarkRed},
    keywords=[2]{assume,predict,infer,observe,define,assume_list,sample,call_back,for},
    keywordstyle=[3]\color{myorange},
    keywords=[3]{if,then,else,lambda,tag,do,proc,repeat,run,pass,true,false,quote,default,all,one,begin,let,lte,letrec,set!},
    keywordstyle=[4]\color{mypurple},
    keywords=[4]{flip,normal,flip,bernoulli,uniform_continuous,uniform_structure,uniform_discrete,gamma,mh,rejection,array,max,length,list,first,second,gridsearch_argmax,apply,add_funcs,mult_funcs,subset,lookup,size,mem\&em,contains,stats,map,mapv,gpmem,quadmem,make_whitenoise, mem,make_gp,make_const_func,make_squaredexp,draw_gp_curves,exclude,closest_point,contents,set_contents,contents_changed?,contents_changed,mark_recently_changed,peek,increment,tail,repeatO,confine,return,print,get_neal_blackbox,get_neal_data_xs,get_data_xs,size,get_bayesopt_blackbox},
    literate=%
    *{0}{{{\color{DarkBlue}0}}}1
    {1}{{{\color{DarkBlue}1}}}1
    {2}{{{\color{DarkBlue}2}}}1
    {3}{{{\color{DarkBlue}3}}}1
    {4}{{{\color{DarkBlue}4}}}1
    {5}{{{\color{DarkBlue}5}}}1
    {6}{{{\color{DarkBlue}6}}}1
    {7}{{{\color{DarkBlue}7}}}1
    {8}{{{\color{DarkBlue}8}}}1
    {9}{{{\color{DarkBlue}9}}}1,
  }
\lstset{
  basicstyle=\singlespacing\ttfamily,
  language=Venture,
  showstringspaces=false,
}

  
% math 
\usepackage{bm}
\usepackage{amsmath,amssymb}
\usepackage{mathtools}

% import new commands, colors and math operators
% math operators
\DeclareMathOperator*{\Cov}{Cov}
\DeclareMathOperator*{\argmax}{arg\,max}
\DeclareMathOperator*{\argmin}{arg\,min}
\DeclareMathOperator{\Uniform}{Uniform}
\DeclareMathOperator{\SE}{SE}

% commands
\newcommand{\gpmem}{\texttt{gpmem}}
\newcommand{\emu}{{\textrm{emu}}}
\newcommand{\true}{{\textrm{true}}}
\newcommand{\rmnew}{{\textrm{new}}}
\newcommand{\past}{{\textrm{past}}}
\newcommand{\prior}{{\textrm{prior}}}
\newcommand{\noisy}{{\textrm{noisy}}}
\newcommand{\noise}{{\textrm{noise}}}
\newcommand{\sigmanoise}{\sigma_{\text{noise}}}
\newcommand{\Acal}{\mathcal{X}}
\newcommand{\R}{\mathbb{R}}
\newcommand{\xprime}{\mathbf{x}^\prime}
\newcommand{\yprime}{\mathbf{y}^\prime}
\newcommand{\yprimetop}{\mathbf{y}^{\prime \top}}
\newcommand{\xstar}{\mathbf{x}^*}
\newcommand{\ystar}{\mathbf{y}^*}
\newcommand{\abf}{\mathbf{x}}
\newcommand{\tbf}{\mathbf{t}}
\newcommand{\fbf}{\mathbf{f}}
\newcommand{\hbf}{\mathbf{h}}
\newcommand{\rbf}{\mathbf{y}}
\newcommand{\wbf}{\mathbf{w}}
\newcommand{\xbf}{\mathbf{x}}
\newcommand{\Mbf}{\mathbf{M}}
\newcommand{\Dbf}{\mathbf{D}}
\newcommand{\ybf}{\mathbf{y}}
\newcommand{\Kbf}{\mathbf{K}}
\newcommand{\Lbf}{\mathbf{L}}
\newcommand{\Sbf}{\mathbf{S}}
\newcommand{\Ktheta}{\mathbf{K}_{\bm{\theta}}}
\newcommand{\ktheta}{k_{\bm{\theta}}}
\newcommand{\Kpost}{\mathbf{K}_{\bm{\theta}}^\text{post}}
\newcommand{\mupost}{\bm{\mu}_{\bm{\theta}}^\text{post}}
\newcommand{\thetabf}{\bm{\theta}}
\newcommand{\Omegabf}{\bm{\Omega}}
\newcommand{\midtheta}{\mid \bm{\theta}}
\newcommand{\Ibf}{\mathbf{I}}
\newcommand{\mubf}{\bm{\mu}}
\newcommand{\Krv}{\bm{\mathcal{K}}}
\newcommand{\Klin}{K^\text{linear}}
\newcommand{\Kper}{K^\text{periodic}}
\newcommand{\klin}{k^\text{linear}}
\newcommand{\kwn}{k^\text{wn}}
\newcommand{\kper}{k^\text{periodic}}
\newcommand{\kse}{k^\text{se}}
\newcommand{\Kse}{K^\text{se}}
\newcommand{\Ksrv}{\bm{\mathcal{K}}}
\newcommand{\Struct}{\text{Struct}}
\newcommand{\Simplify}{\text{Simplify}}
\newcommand{\Parse}{\text{Parse}}
\newcommand{\Cont}{\text{Contains}}
\newcommand{\WN}{\text{WN}}
\newcommand{\WNK}{\text{WN}(\Ksrv)}
\newcommand{\LINK}{\text{LIN}(\Ksrv)}
\newcommand{\PERK}{\text{PER}(\Ksrv)}
\newcommand{\SEK}{\text{SE}(\Ksrv)}
\newcommand{\CK}{\text{C}(\Ksrv)}



\newcommand{\pn}[1]{\left( #1 \right)}
\newcommand{\bkt}[1]{\left[ #1 \right]}
\newcommand{\br}[1]{\left\{ #1 \right\}}
\newcommand{\abs}[1]{\left\lvert #1 \right\rvert}
\newcommand{\Ebkt}[2][]{\mathbb{E}_{#1}\bkt{#2}}
\newcommand{\mvert}{\ \middle\vert\ }

\newcommand{\bmat}[1]{\begin{bmatrix} #1 \end{bmatrix}}

\newcommand{\Ncal}{\mathcal{N}}
\newcommand{\ftt}{\texttt{f}}
\newcommand{\gtt}{\texttt{g}}
\newcommand{\xtt}{\texttt{x}}
\newcommand{\mm}{\texttt{mem\&em}}

\newcommand{\quadmem}{\texttt{quadmem}}
\newcommand{\compute}{{\textrm{compute}}}
\newcommand{\probe}{{\textrm{probe}}}


\newcommand{\rmnext}{{\textrm{next}}}


\newcommand{\noiseless}{{\textrm{noiseless}}}
\newcommand{\proposal}{{\textrm{proposal}}}
\newcommand{\update}{{\textrm{update}}}
\newcommand{\search}{{\textrm{search}}}
\newcommand{\accept}{{\textrm{accept}}}
\newcommand{\current}{{\textrm{current}}}
\newcommand{\MC}{{\textrm{MC}}}
\newcommand{\MH}{{\textrm{MH}}}
\newcommand{\avg}{{\textrm{avg}}}
\newcommand{\reg}{{\textrm{reg}}}
\newcommand{\ttheta}{\vartheta}
\newcommand{\muhat}{\widehat{\mu}}

\newcommand{\xtil}{\widetilde{x}}
\newcommand{\ytil}{\widetilde{y}}

\newcommand{\s}[2]{#2^{(#1)}}
\newcommand{\GP}{\mathcal{GP}}
\newcommand{\propstd}{\texttt{propstd}}



\newcommand{\myparagraph}[1]{\paragraph{#1}\mbox{}\\}

% Colors
\definecolor{mygreen}{rgb}{0,0.6,0}
\definecolor{mygray}{rgb}{0.5,0.5,0.5}
\definecolor{mymauve}{rgb}{0.58,0,0.82}
\definecolor{mygreen}{rgb}{0,0.4,0}
\definecolor{mypurple}{rgb}{0.38,0,0.83}
\definecolor{myorange}{rgb}{0.75,0.3,0}
\definecolor{myblue}{RGB}{76,114,176}






% define acronyms here. Plural of words ending on s can be problematic.
\begin{acronym}
\acro{AAA} {absorb at applications}
\acro{GP} {Gaussian Processes}
\acro{MAP} {Maximum a posteriori}
\acro{MCMC} {Markov Chain Monte Carlo}
\acro{MDP} {Markov Decision Processes} 
\acro{MH} {Metropolis-Hastings}
\acro{PSP} {Primitive Stochastic Prodecure}
\acro{SP} {Stochastic Procedure}
\acro{SPI} {Stochastic Procedure Interface}
\acro{CCF} {Cosmic Calibration Framework}
\end{acronym}




% JMLR stuff - to be edited if accepted.
%\jmlrheading{}{}{}{}{}{}

% Short headings should be running head and authors last names

\ShortHeadings{Probabilistic Programming with Gaussian Processes}{}


\begin{document}
\title{Probabilistic Programming with Gaussian Process Memoization}


\author{\name Ulrich Schaechtle \email ulrich.schaechtle@rhul.ac.uk \\
	      \addr Department of Computer Science\\
              Royal Holloway, University of London
       \AND \name Ben Zinberg \email bzinberg@alum.mit.edu \\
              \addr Computer Science and Artificial Intelligence Laboratory\\
              Massachusetts Institute of Technology
       \AND \name Alexey Radul \email axch@mit.edu \\
              \addr Computer Science and Artificial Intelligence Laboratory\\
              Massachusetts Institute of Technology
       \AND \name Kostas Stathis \email kostas.stathis@rhul.ac.uk\\
              \addr Department of Computer Science\\
       Royal Holloway, University of London
       \AND \name Vikash K. Mansinghka \email vkm@mit.edu \\
	      \addr Computer Science and Artificial Intelligence Laboratory\\
	      Massachusetts Institute of Technology
} 

       \editor{N.A.}

\maketitle
\noindent\begin{abstract}
Gaussian Processes (GPs) are widely used tools in statistics, machine learning, robotics, computer vision, and 
scientific computation. However, despite their popularity, they can be difficult to apply; all but the simplest 
classification or regression applications require specification and inference over complex covariance functions that do 
not admit simple analytical posteriors.  Probabilistic programming shows potential for reducing the latter barrier,
if the computational surface of a GP can be suitably packaged for cooperating with available generic inference tactics.
This paper shows how to embed Gaussian processes in any higher-order 
probabilistic programming language, using an idiom based on memoization, and demonstrates its utility by implementing 
and extending classic and state-of-the-art GP applications. The interface to Gaussian processes, called \gpmem, takes an 
arbitrary real-valued computational process as input and returns a statistical emulator that automatically improves as 
the original process is invoked and its input-output behavior is recorded.  The flexibility of \gpmem\ is illustrated 
via three applications: (i) robust GP regression with hierarchical hyper-parameter learning, (ii) discovering symbolic 
expressions from time-series data by fully Bayesian structure learning over kernels generated by a stochastic grammar, 
and (iii) a bandit formulation of Bayesian optimization with automatic inference and action selection. All applications 
share a single 100-line Python library and require fewer than 20 lines of probabilistic code each.
\end{abstract}

\begin{keywords}
  Probabilistic Programming, Gaussian Processes, Structure Learning, Bayesian Optimization
\end{keywords}


\section{Introduction}
\ac{GP} are widely used tools in statistics~\citep{barry1986nonparametric}, machine learning~\citep{neal1995bayesian,williams1998bayesian,kuss2005assessing,rasmussen2006gaussian,damianou2013deep}, robotics \citep{ferris2006gaussian}, computer vision~\citep{kemmler2013one}, and scientific computation~\citep{kennedy2001bayesian,schneider2008simulations,kwan2013cosmic}.
% ToDo: get a citation of neal on stats!
They are also central to probabilistic numerics, an emerging effort to develop more computationally efficient numerical procedures, and to Bayesian optimization, a family of meta-optimization techniques that are widely used to tune parameters for deep learning algorithms~\citep{snoek2012practical,gelbart2014bayesian}. They have even seen use in artificial intelligence; for example, they provide the key technology behind a project that produces qualitative natural language descriptions of time series~\citep{duvenaud2013structure,lloyd2014automatic}.

This paper describes Gaussian process memoization, a technique for integrating \ac{GP}s into a probabilistic programming language, and demonstrates its utility by re-implementing and extending state-of-the-art applications of the GP. Memoization is a classic programming technique in which a procedure is augmented with an input-output cache that is checked each time before the function is invoked. This prevents unnecessary recomputation, potentially saving time at the cost of increased storage requirements. Gaussian process memoization, implemented by the {\tt gpmem()} procedure, generalizes this idea to include a statistical emulator that uses previously computed values as data in a statistical model that can accurately forecast probable outputs. The covariance function for the Gaussian process is also allowed to be an arbitrary probabilistic program.

This paper presents three applications of gpmem: (i) a replication of results~\citet{neal1997monte} on outlier rejection via hyper-parameter inference; (ii) a fully Bayesian extension to the Automated Statistician project; and (iii) an implementation of Bayesian optimization via Thompson sampling. The first application can in principle be replicated in several other probabilistic languages embedding the proposal that is described in this paper. The remaining two applications rely on distinctive capabilities of Venture~\citep{mansinghka2014venture}: support for fully Bayesian structure learning and language constructs for inference programming. All applications share a single 50-line Python library and require fewer than 20 lines of probabilistic code each.
% ToDo: the 50 lines are bit understated

\subsection{Related Work}
A recent review on artificial intelligence and machine learning in Natur named
\ac{GP}s and probabilistic programming as amongst the most promising directions for 
future research \citep{ghahramani2015probabilistic}. A practitioner can find many software packages
for \ac{GP}s available for free~\citep[e.g.][]{rasmussen2010gaussian,gpy2014,PyGPs}.
However, the context that a practitioner can provide with them is mostly limited to regression and
classification tasks. Yet, \ac{GP}s performed well in applications that go much
further than these tasks. Neal shows an interesting application of Bayesian regression in
the presence of outliers with a hierarchical system of hyper-priors~\citet{neal1997monte}.
\ac{GP}s have been applied to emulated complex computation such as in the case of
\ac{CCF}~\citep{kwan2013cosmic}. At the heart of this project is the
\ac{CCF} that exploits accuracy of large scale  high-resolution cosmological
simulations that are computationally expensive. Here, a sophisticated sampling scheme
provides an optimal sampling strategy for the cosmological models to be
simulated. The measurements from the simulations are then translated into functions that can
be easily interpolated with \ac{GP}s.

The discovery of symbolic relations and concepts has also been the
focus of attention of Bayesian non-parametrics~\citep[e.g.][]{kemp2006learning}.
The Automatic Statistician introduces inductive learning of
symbolic expressions by kernel structure learning, through performing a greedy search over
the space of possible \ac{GP} kernel
structure~\citep{duvenaud2013structure,lloyd2014automatic}. The resulting
structure comes with symbolic interpretation that can be understood by humans. The
engineering involved a significant work-around for the
GPML-toolbox~\citep{rasmussen2010gaussian} which, whilst providing a novel
solution, still struggles to fulfill requirements to be fully Bayesian.

A further example of advanced use of \ac{GP}s beyond regression and
classification is the seemingly deterministic domain of
optimization. To be Bayesian over the values of expensive functions, \ac{GP}s were used
as emulators, leading to efficient optimization of otherwise expensive machine learning algorithms~\citep{snoek2012practical}. Bayesian optimization
updates the belief system of a Bayesian agent when new information about the
true underlying function to optimize becomes available. A sampling scheme for
performing Bayesian Optimization in very high dimensions has been introduced
by~\citep{mahendran2012adaptive}.  


Probabilistic programming systems on the other hand have pushed the boundaries
of models usable for machine learning with general purpose inference machinery.
Examples include picture, which is combining fast data-driven methods for image
interpretation such as deep neural networks with generative
modelling~\citep{kulkarni2015picture}. Classification with large, deep and evolving hierarchical
models as needed for the identification and tracking of resident space objects
was enabled with the Figaro language~\citep{ruttenberg2014hierarchical}.
Nitti and colleagues present a probabilistic programming approach to
planning with \ac{MDP}s in hybrid domains, that is discrete and continuous-valued
domains as well as domains with an unknown number of
objects~\citep{nitti2015planning}. Bayesian Program
induction outperforms deep neural networks in tasks with only one training example provided~\citep{lake2015humanlevel}. This kind of learning is
particularly hard for machine learning while at the same time considerably easy
for humans. 


\setcounter{figure}{0}
\section{Background on Gaussian Processes}
\ac{GP}s are a Bayesian method for regression. We consider the regression input to be real-valued scalars $x_i$ and the regression output as the value of a function $f$ at $x_i$. The complete training data will be denoted by column vectors $\mathbf{x}$ and $\mathbf{f}$. Unseen test data is denoted with $\mathbf{\hat{x}}$ and $\mathbf{\hat{f}}$.
\ac{GP}s present a non-parametric way to express prior knowledge on the space of all possible functions $f$ modeling
a regression relationship.
Formally, a GP is an infinite-dimensional extension of the multivariate Gaussian distribution.
\begin{definition}
A Gaussian process (GP) is a collection of random variables, any
finite number of which have a joint Gaussian distribution~\citep[][chapter 2]{rasmussen2006gaussian}.
\end{definition}
The collection of random variables $\br{f(x)}$ (indexed by $x$) represents the
values of the function $f$ at each location $x$.
We write $f \sim \ac{GP}(m,k)$, where $m$ is the {\em mean function} and $k$ is the {\em covariance function} or {\em kernel}.
That is, $m(x)$ is the prior mean of the random variable $f(x)$, and $k(x,x')$ is the prior covariance of the random variables $f(x)$ and $f(x')$.
Throughout, we write $\mathbf{K}(\xbf,\xbf^\prime)$ for the prior covariance matrix determined by $\xbf$ and $\xbf^\prime$, that is, the covariance between the random vectors $\{f(x)\}_{x \in \xbf}$ and $\{f(x')\}_{x' \in \xbf'}$.

We now compute the predictive posterior distribution of test output $\hat\fbf := f(\hat\xbf)$ conditioned on training data $\fbf := f(\xbf)$.  (Here $\xbf$ and $\hat\xbf$ are known constant vectors, and we are conditioning on an observed value of $\fbf$.)  To simplify the calculation, we will assume the prior mean $m$ is identically zero; once the derivation is done, this assumption can be easily relaxed via translation.

The predictive posterior can be computed by first forming the joint density when both training and test data are treated as randomly chosen from the prior, then fixing the value of $\fbf$ to a constant.  To start, let
\[
  \Sigma := \bmat{
    K(\xbf, \xbf)     & K(\xbf, \hat\xbf)     \\
    K(\hat\xbf, \xbf) & K(\hat\xbf, \hat\xbf)
  }
  \text{ and }
  \Sigma^{-1} =: \bmat{
    M_{11} & M_{12} \\
    M_{21} & M_{22}
  }.
\]
We then have
\[
  P(\fbf, \hat\fbf)
  \propto
  \exp\br{
    -\frac12
    \bmat{\fbf^\top & \hat\fbf^\top}
    \bmat{M_{11} & M_{12} \\ M_{21} & M_{22}}
    \bmat{\fbf \\ \hat\fbf}
  }.
\]
Treating $\fbf$ as a fixed constant, we obtain
\[
  P\pn{\hat\fbf \mvert \fbf}
  \propto
  P(\fbf, \hat\fbf)
  \propto
  \exp\br{
    -\frac12 \hat\fbf^\top M_{22} \hat\fbf
    - \hbf^\top \hat\fbf
  },
\]
where $\hbf = M_{21} \fbf$ is a constant vector.  Thus $P(\hat\fbf | \fbf)$ is Gaussian,
\begin{equation}
  P\pn{\hat\fbf \mvert \fbf} \sim \Ncal(\hat\mubf, \hat\Kbf),
\end{equation}
with covariance matrix $\hat\Kbf = M_{22}^{-1}$.  To find its mean $\hat\mubf$, we note that $P_{\hat\fbf|\fbf}(\hat\fbf + \hat\mubf)$ is Gaussian with the same covariance as $P(\hat\fbf | \fbf)$, but its exponent has no linear term:
\begin{align*}
  P_{\hat\fbf|\fbf} \pn{\hat\fbf + \hat\mubf \mvert \fbf}
  &\propto
  \exp\br{
    -\frac12 (\hat\fbf + \hat\mubf)^\top M_{22} (\hat\fbf + \hat\mubf)
    - \hbf^\top (\hat\fbf + \hat\mubf)
  } \\
  &\propto
  \exp\br{
    -\frac12 \hat\fbf^\top M_{22} \hat\fbf
    - \underbrace{(\hbf + M_{22} \hat\mubf)^\top}_{\text{must be $0$}} \hat\fbf
  }.
\end{align*}
Thus $\hbf = -M_{22} \hat\mubf$ and $\hat\mubf = -M_{22}^{-1} \hbf = -M_{22}^{-1} M_{21} \fbf$.

The partioned inverse equations (\citealp*{barnett1979matrix} following \citealp*{mackay1998introduction}) give
\begin{align*}
  M_{22}^{-1} &= K(\hat\xbf,\hat\xbf) - K(\hat\xbf,\xbf) K(\xbf,\xbf)^{-1} K(\xbf,\hat\xbf), \\
  M_{21} &= -M_{22} K(\hat\xbf,\xbf) K(\xbf,\xbf)^{-1}.
\end{align*}
Substituting these in the above gives
\begin{align}
  \hat\Kbf &= K(\hat\xbf,\hat\xbf) - K(\hat\xbf,\xbf) K(\xbf,\xbf)^{-1} K(\xbf,\hat\xbf), \\
  \hat\mubf &= K(\hat\xbf,\xbf) K(\xbf,\xbf)^{-1}\fbf.
\end{align}

Often one assumes the observed regression output is noisily measured, that is, one only sees the values of $\ybf_\noisy = \mathbf{f}+ \wbf$ where $\wbf$ is Gaussian white noise with variance $\sigma_\noise^2$. This noise term can be absorbed into the covariance matrix $\mathbf{K}(\mathbf{x},\mathbf{x})$ which in the following, we will write as $\mathbf{K}$ for readability. The log-likelihood of a \ac{GP} can then be written as:
\begin{equation}
\label{eq:gplogdens}
\log P(\mathbf{f} \mid \xbf) =
-\frac12 \ybf^\top 
\mathbf{K}^{-1} \ybf
- \frac12\log \abs{\mathbf{K}}
- \frac{n}{2}\log 2\pi
\end{equation}
where $n$ is the number of data points.
Both log-likelihood and predictive posterior can be computed efficiently using a \ac{SP} in Venture~\citep{mansinghka2014venture}
with an algorithm that resorts to Cholesky factorization\citep[chap. 2]{rasmussen2006gaussian}. 
We write the Cholesky factorization as 
$\mathbf{L} \coloneqq \text{chol}(\mathbf{K})$ when
:
\begin{equation}
\mathbf{K} = LL^\top
\end{equation}
where L is a lower triangular matrix. This allows us to compute the inverse of a covariance matrix as
\begin{equation}
\mathbf{K}^{-1} = (\mathbf{L}^{-1})^\top (\mathbf{L}^{-1})
\end{equation}
and its determinant as 
\begin{equation}
det(\mathbf{K}) = det(\mathbf{L})^2
\end{equation}
We compute (\ref{eq:gplogdens}) as
\begin{equation}
\log(P(\mathbf{f}\mid \mathbf{x})\coloneqq - \frac{1}{2} \mathbf{f}^\top \bm{\alpha} - \sum_i \log \mathbf{L}_{ii} - \frac{n}{2} \log 2 \pi
\end{equation}
where 
\begin{equation}
\label{eq:chol_L}
\mathbf{L} \coloneqq \text{chol}(\mathbf{K})
\end{equation}
and 
\begin{equation}
\label{eq:alpha}
\bm{\alpha} \coloneqq  \mathbf{L}^\top \backslash(\mathbf{L} \backslash \mathbf{f}). 
\end{equation}
This results in a computational complexity of $\mathcal{O}(n^3)$ in the number of data points for
sampling with a complexity of $n^3/6$ for (\ref{eq:chol_L}) an $n^2/2$ for (\ref{eq:alpha}). 



The covariance function (or kernel) of a \ac{GP} governs high-level properties of the observed data such as linearity, periodicity and smoothness.
It comes with few free parameters that we call hyper-parameters.
Adjusting the hyper-parameters changes non-qualitative attributes such as length
scales while preserving the qualitative properties of the distribution.
These high-level properties are compositional via addition and multiplication of different covariance functions. An addition models a global interaction, that is an interaction of two high-level components that is qualitatively not dependent on the input space. An example for this a periodic function with a linear trend.
A multiplication models a local interaction of two components. 
An example for this is a periodic function with a linearly increasing amplitude. We demonstrate kernel composition with local interactions in the tutorial in Fig. \ref{fig:composition_tutorial}. 


\begin{figure}
 \centering
     \begin{subfigure}[b]{0.3\textwidth}
        \includegraphics[width=\textwidth]{figs/composition/composition_demo_LIN_prior.png}
        \caption{LIN}
    \end{subfigure}
    ~ %add desired spacing between images, e. g. ~, \quad, \qquad, \hfill etc. 
      %(or a blank line to force the subfigure onto a new line)
    \begin{subfigure}[b]{0.3\textwidth}
        \includegraphics[width=\textwidth]{figs/composition/composition_demo_PER_prior.png}
        \caption{PER}
    \end{subfigure}
    ~ %add desired spacing between images, e. g. ~, \quad, \qquad, \hfill etc. 
    %(or a blank line to force the subfigure onto a new line)
    \begin{subfigure}[b]{0.3\textwidth}
        \includegraphics[width=\textwidth]{figs/composition/composition_demo_LINxPER_prior.png}
        \caption{LIN $\times$ PER}

    \end{subfigure}   \begin{subfigure}[b]{0.3\textwidth}
        \includegraphics[width=\textwidth]{figs/composition/composition_demo_LIN.png}
        \caption{LIN}
    \end{subfigure}
    ~ %add desired spacing between images, e. g. ~, \quad, \qquad, \hfill etc. 
      %(or a blank line to force the subfigure onto a new line)
    \begin{subfigure}[b]{0.3\textwidth}
        \includegraphics[width=\textwidth]{figs/composition/composition_demo_PER.png}
        \caption{PER}
    \end{subfigure}
    ~ %add desired spacing between images, e. g. ~, \quad, \qquad, \hfill etc. 
    %(or a blank line to force the subfigure onto a new line)
    \begin{subfigure}[b]{0.3\textwidth}
        \includegraphics[width=\textwidth]{figs/composition/composition_demo_LINxPER.png}
        \caption{LIN $\times$ PER}
    \end{subfigure}

%20.0739791735
%6.31647597198
%37.7184218042
%19.1051376016

\caption{We depict kernel composition. 
(a) shows raw data (black) generated with a sine function with linearly growing amplitude (blue). 
(b) shows the linear and the periodic base kernel as well as a composition of both. 
The multiplication of the two kernels indicates local interaction. The local interaction we account for in this case is the growing amplitude (a). (c-e) show the parameterized kernels that we introduced in (b).
The parameters are sampled from the posterior distribution on parameters for a \ac{GP} with kernel $\mathbf{K}=\text{LIN} \times \text{PER}$ where the data points from (a) are observed.
(f-h) show samples from the prior where the parametrization from (c-e) is used, that is, before any data points are observed.
(i-k) show samples from the posterior, after the data has been observed. Note that the same data is used for all of the plots.}
\label{fig:composition_tutorial}
\end{figure}





Venture includes the primitive \texttt{make\_gp}, which takes as arguments a
unary function \texttt{mean} and a binary (symmetric, positive-semidefinite)
function \texttt{cov} and produces a function $\gtt$ distributed as a Gaussian
process with the supplied mean and covariance.  For example, a function $\gtt
\sim \GP(0,\,\SE)$, where $\SE$ is a squared-exponential covariance
\[ \SE(x, x') = \sigma^2 \exp\pn{\frac{(x-x')^2}{2\ell}} \]
with $\sigma=1$ and $\ell=1$, can be instantiated as follows:
\begin{lstlisting}[language=Venture]
assume zero = make_const_func( 0.0)
assume se = make_squaredexp( 1.0, 1.0)
assume g  = make_gp( zero, se)
\end{lstlisting}
There are two ways two view $\gtt$ as a ``random function.'' In the first view,
the \texttt{assume} directive that instantiates $\gtt$ does not use any
randomness---only the subsequent calls to $\gtt$ do---and coherence constraints
are upheld by the interpreter by keeping track of which evaluations of $\gtt$
exist in the current execution trace.  Namely, if the current trace contains evaluations
of $\gtt$ at the points $x_1,\ldots,x_N$ with return values $y_1,\ldots,y_N$,
then the next evaluation of $\gtt$ (say, jointly at the points $x_{N+1}, \ldots,
x_{N+n}$) will be distributed according to the joint conditional distribution
\[
  P\pn{\big.
    \texttt(\gtt\ x_{N+1}\texttt), \ldots, \texttt(\gtt\ x_{N+n}\texttt)
    \mvert
    \texttt(\gtt\ x_i\texttt) = f_i \text{ for $i=1,\ldots,N$}}.
\]
In the second view, $\gtt$ is a randomly chosen deterministic function, chosen
from the space of all deterministic real-valued functions; in this view, the
\texttt{assume} directive contains \emph{all} the randomness, and subsequent
invocations of $\gtt$ are deterministic.  The first view is procedural and is
faithful to the computation that occurs behind the scenes in Venture.  The
second view is declarative and is faithful to notations like ``$g \sim P(g)$''
which are often used in mathematical treatments.  Because a model program could
make arbitrarily many calls to $\gtt$, and the joint distribution on the return
values of the calls could have arbitrarily high entropy, it is not
computationally possible in finite time to choose the entire function $\gtt$ all
at once as in the second view.  Thus, it stands to reason that any
computationally implementable notion of ``nonparametric random functions'' must
involve incremental random choices in one way or another, and \ac{GP}s
in Venture are no exception.


\subsection{Packaging Gaussian Processes for Venture}
Random samples from a \ac{GP}, like all random samples in Venture programs,
are generated with the invocation of \ac{SP}s. \ac{SP}s are the basic units of
computation in Venture. They provide a general framework to think about
probabilities that is particularly handy for \ac{GP}s. This framework does not
only define the implementation of a \ac{GP} model and possible inference over it but also
the representation and formulation of a \ac{GP} as a probabilistic model.
This aspect is somewhat different from the view of traditional machine learning,
seeing the software engineering and programming aspects only as a vehicle to
realize and communicate an idea.  In contrast, the language of samplers and \ac{SP}s provide a
different view point to what a \ac{GP} can and cannot compute.

\ac{SP}s in general accept input arguments that are values in Venture and sample output
values given those inputs. They can be deterministic or random. If an \ac{SP} is random
then it can simulate from a family of distributions. In addition to simulating,
a random \ac{SP} may be able to report the log-density of an output given an input.
Different \ac{SP}s are used to construct the \ac{GP} interface:
\begin{enumerate}
  \item MakeGPOutputSP is a deterministic \ac{SP} whose output is a
    \ac{GP}-sampler, namely the GPSP, given a covariance and a mean function as input. 
\item GPSP of is type SP\@. GPSP is responsible for tracking the sufficient statistics
from the applications of GPSP and for evaluating the log density of all those
applications as a block.
\item GPOutputPSP of type Random PSP\@.  It is this SP that actually samples
regression output at given input.
\end{enumerate}


%procedures that return other stochastic procedures and implement this optimization are said to be
%absorbing at applications, often abbreviated AAA.




\section{Gaussian Process Memoization in Venture}
Memoization is the practice of storing previously computed values of a function so that future calls with the same inputs can be evaluated by lookup rather than re-computation.
To transfer this idea to probabilistic programming, we now introduce a language construct called a
\emph{statistical memoizer}.  Suppose we have a function $\ftt$ which can be evaluated 
but we wish to learn about the behavior of $\ftt$ using as
few evaluations as possible.  The statistical memoizer, which here we give the
name \gpmem, was motivated by this purpose.  It produces two outputs:
\[ \ftt \xrightarrow{\gpmem} (\ftt_{\text{compute}}, \ftt_\emu). \]
The function $\ftt_{compute}$ calls $\ftt$ and stores the output in a memo
table, just as traditional memoization does.  The function $\ftt_\emu$ is
an online statistical emulator which uses the memo table as its training
data.  A fully Bayesian emulator, modelling the true function $\ftt$ as a
random function $f \sim P(f)$, would satisfy
\[
\texttt{(}\ftt_\emu\ \xtt_1\ \ldots\ \xtt_k\texttt{)}
\sim
P\pn{
  f(\xtt_1), \ldots, f(\xtt_k)
  \mvert
  \text{$f(\xtt) = \texttt{(f x)}$ for each $\xtt$ in memo table}
}.
\]
Different implementations of the statistical memoizer can have
different prior distributions $P(f)$; in this paper, we deploy a \ac{GP} 
prior (implemented as \texttt{gpmem} below).  Note that we require the ability
to sample $\ftt_\emu$ jointly at multiple inputs because the values of
$f(\xtt_1),\ldots,f(\xtt_k)$ will in general be dependent.


\begin{figure}
\centering
\begin{tabular}{ll}\small
% line 1
\begin{lstlisting}[mathescape,escapechar=\#]
assume (f_compute f_emu) =  gpmem( f)
sample (f_emu( array( -20, $\cdots$, 20)) 

             #\raisebox{-0.2\height}{\includegraphics[height=.5cm]{figs/oneline.png}}# $\sim \mathcal{N}(\mu,\mathbf{K})$
\end{lstlisting}
& \raisebox{-0.5\height}{\includegraphics[height=2.5cm]{figs/slide1_0pred.png}} \\ \hline
% line 2
\begin{lstlisting}[mathescape,escapechar=\#]
predict f_compute( 12.6)

sample (f_emu( array( -20, $\cdots$, 20)) 

\end{lstlisting}
 &  \raisebox{-0.5\height}{\includegraphics[height=2.5cm]{figs/slide1_1pred.png}}  \\ \hline
% line 3
 \begin{lstlisting}[mathescape,escapechar=\#]
predict f_compute( -6.4)

sample (f_emu( array( -20, $\cdots$, 20)) 
  
\end{lstlisting}
 &   \raisebox{-0.5\height}{\includegraphics[height=2.5cm]{figs/slide1_2pred.png}}
\end{tabular}
%\put(-245,10){\line(1,0){200}}
\put(-33,77){\color{ForestGreen}\thicklines \vector(0,-1){15}}
\put(-96,6){\color{ForestGreen}\thicklines \vector(0,-1){15}}
\put(-48,-63){\thicklines \vector(0,-1){15}}
\put(-84,-43){\thicklines \vector(0,-1){15}}
\put(-73,-68){\thicklines \vector(0,1){15}}
\put(-460,110){(b)}
\put(-460,45){(c)}
\put(-460,-10){(d)}
\put(-460,-75){(e)}
\put(-460,-130){(f)}
\caption{\small \gpmem\ tutorial. The top shows a schematic of \gpmem.
  \texttt{f\_compute} probes an outside resource.
  This can be expensive (top left).
  Every probe is memoized and improves the emulator. Below the schematic we see the evolution
  of \gpmem's state of believe of the world given certain Venture
  directives. On the right, we depict the true function (blue), samples from the emulator (red) and incorporated observations (black).}
\label{fig:gpmem_tutorial}
\end{figure}

% Panel 1
We explain how \gpmem, the statistical memoizer with \ac{GP}-prior, works using a simple tutorial
(Fig. \ref{fig:gpmem_tutorial}). 
The top panel (Fig. \ref{fig:gpmem_tutorial}, (a)) of this figure sketches the schematic of \gpmem.
$\ftt$ is the external process that we memoize. It can be evaluated using resources that potentially come
from outside of Venture.  
We feed this function into \gpmem\ alongside
a parameterised kernel $\Ktheta$.  
In this example, we make the qualitative assumption of $f$ being smooth, and define
$\Ktheta$ to be a squared-exponential covariance function:
\[
\Ktheta = \text{SE} = \sigma^2 \exp(-\frac{(x-x^\prime)^2}{2\ell^2}).
\]
The hyper-parameters $\thetabf$ for this kernel are sampled from a 
prior distribution which is depicted in the top right box.
Note that we annotate $\bm{\theta}=\{\texttt{sf},\texttt{l}\}$ for subsequent
inference as belonging to (i) the scope "hyper-parameter" and (ii) blocks 0 and 1 respectively.

\gpmem\ implements a memoization table, where all previously
computed function evaluations are stored. We also initialize a \ac{GP}-prior that
will serve as our statistical emulator:
\[
P(f_{emu}(x) \mid \mathbf{x}_{past},\fbf)\sim \mathcal{N}(\mu(\mathbf{x}),\mathbf{K}_\theta\big(\mathbf{x},\mathbf{x})\big)
\]
where 
\[
P(f_{emu}(x) \mid \mathbf{x}_{past},\fbf) = \hat\fbf 
\]
under the traditional \ac{GP} perspective\footnote{Note that we write $\fbf_*$ instead of $\hat\fbf$ if $\xbf_{past}=\{\}$ and $\fbf=\{\}$.}.
All value pairs stored in the memoization table ($\text{memo table} = (\mathbf{x}_{past},\fbf)$) are incorporated as observations of
the \ac{GP}.
The emulator allows us to make probabilistic predictions on function evaluations, estimating
$P(\hat\fbf\mid \fbf)$ with $\mathcal{N}(\hat{\bm{\mu}},\hat\Kbf)$.
We simply feed the regression input
into the emulator and output a predictive posterior Gaussian distribution determined by the \ac{GP} and
the memoization table.

% Panel 2
We can either define the function f that serves as as input for \gpmem\
 natively in Venture
(as shown in the Fig. \ref{fig:gpmem_tutorial} (b)) or we interleave Venture with foreign code. 
This can be useful when $\ftt$ is computed with the help of outside resources.
Before making any observations or calls to $\ftt$
we can sample from the prior at the inputs from -20 to 20 using the emulator:
    \begin{lstlisting}
    assume (f_compute, f_emu) = gpmem(f, K))

    sample f_emu(array(-20, ..., 20))
    \end{lstlisting}
where the second line corresponds to:
\[ 
\fbf_* \sim \mathcal{N}\Bigg(0,K\bigg(
\bmat{
-20 \\
\cdots \\
20
},
\bmat{
-20 \\
\cdots \\
20
}
\bigg)
\Bigg).
\]
% Panel 3
In Fig. \ref{fig:gpmem_tutorial} (c), we probe the external function $\ftt$ at point 12.6 and memoize it's result by calling 
   \begin{lstlisting}
    predict f_compute (12.6).
    \end{lstlisting}
When we subsequently sample from the emulator, that is compute the $\hat\fbf$ at the input
$\hat\xbf= \bmat{-20, \cdots, 20}^\top$. We see how the posterior which shifts from uncertainty to near certainty close to the input 12.6.

% Panel 4
We can repeat the process at a different point (probing point -6.4 in Fig.
\ref{fig:gpmem_tutorial} (d)) to see that we gain certainty about another part of the curve. 

% Panel 5
We can add information to $\texttt{f}_\text{emu}$ about presumable value pairs of $\ftt$ without calling $\texttt{f}_\text{compute}$
(Fig. \ref{fig:gpmem_tutorial} (e)).
If a friend tells us the value of $\ftt$ we can call observe to store this information in the incorporated observations for $\texttt{f}_\text{emu}$ only:
    \begin{lstlisting}
    observe f_emu( -3.1) = 2.60.
    \end{lstlisting}
We have this value pair now available for the computation $\hat\fbf$. 
For sampling with the emulator, the effect is the same as calling predict with the $\texttt{f}_\text{compute}$.
However, we can imagine at least one scenario where such as distinction in the treatment of observations 
is beneficial. Let us say we do not only have the real function available but also a domain expert with knowledge 
about this function.
This expert could tell us what the value is at a given input.
Potentially, the value provided by the expert could disagree with the value computed with $\ftt$ for example 
due to different levels of observation noise. 

% Panel 6
Finally, we can update our posterior by inferring the posterior over hyper-parameter values $\thetabf$.
We can take 50 \ac{MH} steps by calling that where we take an \ac{MH} steps for either of the two blocks for the scope "hyper-parameter": 
   \begin{lstlisting}
    infer mh( 'hyper-parameter, one, 50).
    \end{lstlisting}
The newly inferred hyper-parameters allow us now adequately reflect uncertainty
about the curve given all incorporated observations (compare
Fig. \ref{fig:gpmem_tutorial}, bottom panel (f) on  the right with the samples
before inference, one panel above (e)).


\section{Applications}
This paper illustrates the flexibility of \gpmem\ by showing how it can concisely encode three different applications of \ac{GP}s.
The first is a standard example from hierarchical Bayesian statistics, where Bayesian inference over a hierarchical hyper-prior is used to provide a curve-fitting methodology that is robust to outliers.
The second is a structure learning application from probabilistic artificial intelligence, where \ac{GP}s are used to discover qualitative structure in time series data.
The third is a reinforcement learning application, where \ac{GP}s are used as part of a Thompson sampling formulation of Bayesian optimization for general real-valued objective functions with real inputs.

\subsection{Nonlinear regression in the presence of outliers}
We can apply \gpmem\ for regression in a hierarchical Bayesian setting
(Fig. \ref{fig:neal_tutorial}).  
%%%%%%%%%%%%%%%%%%%%%%%%%%%%%%%%%%%%%%%%%%%%%%%%%%%%%%%%%%%%
\begin{figure}
\renewcommand{\arraystretch}{0.1}% Tighter
\centering \footnotesize
\begin{tikzpicture}
\node[] (start) {};
\node[left=1cm of start] (sfsigma) {\includegraphics[height=2cm]{figs/hypers_sf_sigma.png}};
\node[right=1cm of start] (sfell) {\includegraphics[height=2cm]{figs/hypers_sf_ell.png}};
\node[above=1cm of start] (hyper) {(a) $P(\bm{\theta} = \{\ell,sf,\sigma\} \mid
\mathbf{D},\Krv)$};
\node[left=0.cm of sfsigma] (ell) {$\ell$};
\node[below=0.cm of sfsigma] (sf1) {$sf$};
\node[left=0.cm of sfell] (sigma) {$\sigma$};
\node[below=0.cm of sfell] (sf2) {$sf$};






%%%%%%%%%%%%%%%%%%%%%%%%%%%%%%%%%%%%%%%%%%%%%%%%%%%%%%%%%%%%
%%%%%%%%%%%%  Code             %%%%%%%%%%%%%%%%%%%%%%%%%%%%%
%%%%%%%%%%%%%%%%%%%%%%%%%%%%%%%%%%%%%%%%%%%%%%%%%%%%%%%%%%%%

\node[below=1.0cm of start] (b_code){
\footnotesize\begin{lstlisting}[mathescape,escapechar=\@]
// Define data and look-up function
define data      = array(array(-1.87, 0.13),@\ldots@, array(1.67,0.81));
assume f_look_up = proc(index) {lookup(data, index)};
\end{lstlisting}
};
\node[below=-0.65cm of b_code,xshift=-0.58cm] (c_code){
\footnotesize\begin{lstlisting}[mathescape,escapechar=\@
]
// Initialize hyper-priors
assume alpha_sf @$\sim$@ gamma(5,1)                 #"hyperhyper";
assume beta_sf  @$\sim$@ gamma(5,1)                 #"hyperhyper";
assume alpha_l  @$\sim$@ gamma(5,1)                 #"hyperhyper";
assume beta_l   @$\sim$@ gamma(5,1)                 #"hyperhyper";@\vspace{1mm}@
assume sf       @$\sim$@ gamma(alpha_sf, beta_sf))) #"hyper";
assume l        @$\sim$@ gamma(alpha_l, beta_l)))   #"hyper";
assume sigma    @$\sim$@ gamma(5,1))                #"hyper"; 
\end{lstlisting}
};
% (d) & (e)
\node[below=-0.65cm of c_code,xshift=-0.65cm] (d_code){
\footnotesize\begin{lstlisting}[mathescape,escapechar=\@]
// Initialize covariance function
assume se = @\se@;
assume wn = @\wn@;
assume composite_covariance = gp_cov_sum(se, wn);
// Create a prober and emulator using gpmem
assume (f_compute, f_emu) =
    gpmem(f_look_up, composite_covariance);@\vspace{1mm}@
sample f_emu(array(-2, $\cdots$, 2));
\end{lstlisting}
};
% (f)
\node[below=0.16cm of d_code,xshift=-0.18cm] (e_code){
\footnotesize\begin{lstlisting}[mathescape,escapechar=\@]

// Observe all data points
for (x,y) in data {
                   observe f_emu(x) = y};
// Or: probe all data points
for (x,_) in data {
                   predict f_compute(x)};@\vspace{1mm}@
sample f_emu(array(-2, $\cdots$, 2));
\end{lstlisting}
};
% (g)
\node[below=-0.0cm of e_code,xshift=-0.77cm] (f_code){
\footnotesize\begin{lstlisting}[mathescape,escapechar=\@]
// Metropolis-Hastings
infer repeat(100, do(
                mh(#"hyperhyper", steps=1),
                mh(#"hyper",      steps=3)));@\vspace{1mm}@
sample f_emu(array(-2, $\cdots$, 2));
\end{lstlisting}
};

\node[below=0.5cm of f_code,xshift=0.3cm] (g_code){
\footnotesize\begin{lstlisting}[mathescape,escapechar=\@]
// Optimization
infer gradient-ascent(#"hyper", steps=10);

sample f_emu(array(-2, $\cdots$, 2));
\end{lstlisting}
};



% samples/curve images
\node[below=1.1cm of c_code,xshift=6cm] (e_pic){
\includegraphics[height=3.4cm]{figs/neal_example_before_observation.jpg}
};
\node[below=-0.4cm of e_pic] (f_pic){
\includegraphics[height=3.4cm]{figs/neal_example_after_observation.jpg}
};
\node[below=-0.4cm of f_pic] (g_pic){
\includegraphics[height=3.4cm]{figs/neal_Bayesian.jpg}
};
\node[below=-0.4cm of g_pic] (h_pic){
\includegraphics[height=3.4cm]{figs/neal_example_map_inference_alpha0p01_iter15.jpg}
};

% horizontal lines
% (a)
\draw 
  ([xshift=-0.2cm,yshift=-0.6cm]b_code.north west) --
([xshift=0.8cm,yshift=-0.6cm]b_code.north east); 
% (b)
\draw 
  ([xshift=-0.2cm,yshift=0.08cm]b_code.south west) -- ([xshift=0.8cm,yshift=0.08cm]b_code.south east);   
% (c)
 \draw 
  ([xshift=-0.2cm,yshift=-3.17cm]b_code.south west) --
([xshift=0.8cm,yshift=-3.17cm]b_code.south east); 
% (d)
 \draw 
  ([xshift=-0.2cm,yshift=-4.78cm]b_code.south west) --
([xshift=0.8cm,yshift=-4.78cm]b_code.south east); 
% (e)
 \draw 
  ([xshift=-0.2cm,yshift=-7.7cm]b_code.south west) -- ([xshift=0.8cm,yshift=-7.7cm]b_code.south east); 
% (f)
 \draw 
  ([xshift=-0.2cm,yshift=-10.8cm]b_code.south west) -- ([xshift=0.8cm,yshift=-10.8cm]b_code.south east); 
% (g)
 \draw 
  ([xshift=-0.2cm,yshift=-14.1cm]b_code.south west) -- ([xshift=0.8cm,yshift=-14.1cm]b_code.south east); 

% labels
\node[below=1.9cm of start,xshift=-6.5cm] (b) {(b)};
\node[below=1.4cm of b] (c) {(c)};
\node[below=1.9cm of c] (d) {(d)};
\node[below=1.7cm of d] (e) {(e)};
\node[below=2.6cm of e] (f) {(f)};
\node[below=2.9cm of f] (g) {(g)};
\node[below=2.6cm of g] (h) {(h)};
\end{tikzpicture}


\put(-410,230){(b)}
\put(-410,170){(c)}
\put(-410,90){(d)}
\put(-410,-10){(e)}
\put(-410,-110){(f)}
\put(-410,-210){(g)}
\caption{Regression with outliers and hierarchical prior structure}
\label{fig:neal_tutorial}
\end{figure}
%%%%%%%%%%%%%%%%%%%%%%%%%%%%%%%%%%%%%%%%%%%%%%%%%%%%%%%%%%%%
In a Bayesian treatment of  hyper-parameter learning for \ac{GP}s,
we can write the posterior probability of the hyper-parameters of a GP  (Fig.
\ref{fig:neal_tutorial}, (a)) given covariance function $\mathbf{K}$ as:
%%%%%%%%%%%%%%%%%%%%%%%%%%%%%%%%%%%%%%%%%%%%%%%%%%%%%%%%%%%%
\begin{equation}
\label{eq:hyperProbability}
P(\bm{\theta}=\{sf,\ell,\sigma\} \mid \mathbf{D,K}) = \frac{P(\mathbf{D} \mid \bm{\theta}, \mathbf{K})P(\bm{\theta} \mid  \mathbf{K})}{P(\mathbf{D} \mid \mathbf{K})}
\end{equation}
%%%%%%%%%%%%%%%%%%%%%%%%%%%%%%%%%%%%%%%%%%%%%%%%%%%%%%%%%%%%
where $\mathbf{D} = \{\xbf, \fbf\}$ is a training data set. Since we can apply
\gpmem\ to any process or procedure, it can be used in situations where only a data
set is available via a look-up function $\texttt{f\_look\_up}$.
In fact, we demonstrate \gpmem's application to regression using an example where
the data was generated by a function which is not available (Fig.
\ref{fig:neal_tutorial} (b)).
This function, $f_\text{true}$, is taken from a paper on the
treatment of outliers with hierarchical Bayesian hyper-priors for
\ac{GP}s~\citep{neal1997monte}:
%%%%%%%%%%%%%%%%%%%%%%%%%%%%%%%%%%%%%%%%%%%%%%%%%%%%%%%%%%%%
\begin{equation}
f_\text{true}(x) =  0.3 + 0.4 x + 0.5 \sin(2.7x) + \frac{1.1}{(1+ x^2)} + \eta
\;\;\; with\;\;\eta \sim \mathcal{N}(0,\sigma).
\end{equation}
%%%%%%%%%%%%%%%%%%%%%%%%%%%%%%%%%%%%%%%%%%%%%%%%%%%%%%%%%%%%
We synthetically generate outliers by setting $\sigma = 0.1$ in $95\%$ of the cases and to $\sigma = 1$ in the remaining cases. 
Instead of accessing the $f_\text{true}$ directly, we are accessing the $\texttt{data}$ in form of
a a two dimensional $\texttt{array}$ with $\texttt{f\_look\_up}$.

We parameterize $\Kbf_{\bm{\theta}}$ with $\bm{\theta}=\{sf,\ell,\sigma\}$.
For these hyper-parameters, Neals work suggests a hierarchical system for
for hyper-parameterization.
Here, we draw hyper-parameters from $\Gamma$ distributions:
%%%%%%%%%%%%%%%%%%%%%%%%%%%%%%%%%%%%%%%%%%%%%%%%%%%%%%%%%%%%
\begin{equation}
\label{eq:hyper-ell}
\ell^{(t)} \sim \Gamma(\alpha_1,\beta_1),\;\sigma^{(t)} \sim \Gamma(\alpha_2,\beta_2)
\end{equation} 
%%%%%%%%%%%%%%%%%%%%%%%%%%%%%%%%%%%%%%%%%%%%%%%%%%%%%%%%%%%%
and in turn sample the $\alpha$ and $\beta$ from $\Gamma$ distributions as well:
%%%%%%%%%%%%%%%%%%%%%%%%%%%%%%%%%%%%%%%%%%%%%%%%%%%%%%%%%%%%
\begin{equation}
\label{eq:hyper-alpha}
\alpha_1^{(t)} \sim \Gamma(\alpha^1_{\alpha},\beta^1_{ \alpha } ),\; \alpha_2^{(t)} \sim \Gamma(\alpha^2_{\alpha},\beta^2_{\alpha}),\cdots
\end{equation}
%%%%%%%%%%%%%%%%%%%%%%%%%%%%%%%%%%%%%%%%%%%%%%%%%%%%%%%%%%%%
We model this in Venture as illustrated in Fig. \ref{fig:neal_tutorial} (c),
using the build-in \ac{SP} $\texttt{gamma}$. Note that we omit the prior distributions of (\ref{eq:hyper-alpha}) due limited space in the tutorial.

In Fig. \ref{fig:neal_tutorial}, panel (d), we see that $\mathbf{K}_{\bm{\theta}}$
is defined as a composite covariance function. It is the sum ($\texttt{add\_funcs}$) of
an SE kernel ($\texttt{make\_squaredexp}$) and a white noise (WN, Appendix A)
kernel which is implemented with $\texttt{make\_whitenoise}$\footnote{Note
that in Neal's work \citeyearpar{neal1997monte} the sum of an SE
plus a constant kernel is used. We use a WN kernel for illustrative purposes
instead.}. 
We then initialize \gpmem\ feeding it with $\texttt{composite\_covariance}$ and the data
look-up function $\texttt{f\_look\_up}$. 
We sample from the prior $\fbf_*$ with random parameters $\texttt{sf,l}$ and $\texttt{sigma}$ and 
without any observations available.
We depict those samples on the right (red), alongside the true function that generated the data (blue) and
the data points we have available in the data set (black).

We can incorporate observations using both \texttt{observe} and \texttt{predict} (Fig. \ref{fig:neal_tutorial} (e)).
When we subsequently sample $\hat\fbf$ from the emulator with
$\mathcal{N}(\hat{\bm{\mu}},\hat\Kbf)$, we can see that the \ac{GP} posterior incorporates knowledge 
about the $\texttt{data}$. Yet, the hyper-parameters $\texttt{sf,l}$ and $\texttt{sigma}$ are still
random, so the emulator does not capture the true underlying dynamics
($f_\text{true}$) of the \texttt{data} correctly. 

Next, we demonstrate how we can capture these underlying dynamics within only
100  nested \ac{MH} steps on the hyper-parameters to get a good approximation
for their posterior $\hat\fbf$ (Fig. \ref{fig:neal_tutorial} (f)).
We say nested because we first take two sweeps in the scope
$\texttt{hyperhyper}$ which characterizes (\ref{eq:hyper-alpha}) and then one
sweep on the scope $\texttt{hyper}$ which characterizes (\ref{eq:hyper-ell}).
This is repeated 100 using $\texttt{repeat( 100, do(}\cdots\;$.
Note that Neal devises an additional noise model and performs a large number of Hybrid-Monte Carlo and Gibbs steps to achieve this, whereas inference in Venture with \gpmem\ is merely one line of code. 

Finally, we can change our inference strategy altogether. If we decide that instead of
following a Bayesian sampling approach, we would like to perform empirical optimization,
we do this by only changing one line of code, deploying $\texttt{map}$ instead
of $\texttt{mh}$ (Fig. \ref{fig:neal_tutorial} (g)). 


\subsection{Discovering qualitative structure from time series data}\label{sec:structurelearning}
Inductive learning of symbolic expressions for continuous-valued time series
data is a hard task which has recently been tackled using a greedy search over 
the approximate posterior of the possible kernel compositions for
\ac{GP}s~\citep{duvenaud2013structure,lloyd2014automatic}\footnote{\url{http://www.automaticstatistician.com/}}.

With \gpmem\ we can provide a fully Bayesian treatment of this, previously unavaible,
using a stochastic grammar  (see Fig. \ref{fig:schema}).
%%%%%%%%%%%%%%%%%%%%%%%%%%%%%%%%%%%%%%%%%%%%%%%%%%%%%%%%%%%%
\begin{figure}
\centering
\usetikzlibrary{arrows, decorations.markings}
\usetikzlibrary{trees}

\tikzstyle{level 1}=[level distance=1cm, sibling distance=1.5cm]
\tikzstyle{level 2}=[level distance=1cm, sibling distance=1.1cm]

% Define styles for operators and leafs
\tikzstyle{operator} = [draw=none,circle, minimum width=1pt]
\tikzstyle{end} = [circle, minimum width=3pt,fill, inner sep=0pt]
% for double arrows a la chef
% adapt line thickness and line width, if needed

\begin{subfigure}[b]{0.49\textwidth}\centering
\begin{tikzpicture}[thick]

 \node[] (base_kernels) {\small$\text{BK}=\{k_{\bm{\theta}^1}^1,\cdots,k_{\bm{\theta}^m}^m\}$};
 \node[above=0.7cm of base_kernels] (theta) {\small$\bm{\theta}^* \sim P(\bm{\theta}^*)$};
 \node[draw,rectangle, below=1cm of start, text width =6.0cm, text 
height=4.1cm,align=center] (grammar)
{};


 \node[draw,dashed,rectangle, below=-3.8cm of grammar, text width
=5.3cm,align=center] (subset) {
$\;$Select Primitive Kernels\\% I can't believe that this is the simpliest way to
% get spacing right 
{\raggedright
\footnotesize$\Sbf  \sim P(\Sbf =
\{k_{\bm{\theta}^i}^i,\cdots,k_{\bm{\theta}^n}^n\} \mid \text{BK})$
}
};
 \node[draw,rectangle,dashed, below=0.7cm of subset, text width
=4.35cm,align=center]
(composition_procedure) {
$\;$Kernel Composer\\ % I can't believe that this is the simpliest way to
% get spacing right 
{\raggedright
\footnotesize$\;\;\;\bm{\Omega} \sim P(\bm{\Omega} \mid \Sbf)$ \\

\footnotesize$k_{\bm{\theta}} \sim P(\Krv \mid \bm{\Omega},\Sbf),\;\,\bm{\theta}\subseteq\bm{\theta}^*$
}
};

\node[above=0.0cm of subset]{\centering \bf Stochastic Grammar}; 

 \node[draw,rectangle,below=1cm of grammar] (gpmem) {\texttt{gpmem}};

 \node[draw,rectangle,below=1cm of gpmem] (f) {Data Generation};
\node[left of=f,xshift=-1.5cm] (x) {$\xbf$};
\node[right of=f,xshift=1.5cm] (y) {$\ybf$};

 \node[below=0.1cm  of grammar,xshift=0.5cm] (k)
{\small $k_{\thetabf}$};


\node[below=0.1cm  of gpmem,xshift=0.6cm] (gp) {
\small$f_{emu}$};

% 1st pass: draw arrows
  \draw[thick,->] (base_kernels) -- (grammar);
 
  \draw[thick,->] (theta) -- (base_kernels);
  \draw[thick,->] (grammar) -- (gpmem);
  \draw[thick,->] (gpmem) -- (f);
 \draw[thick,->] (x) -- (f);
 \draw[thick,->] (f) -- (y);
 \draw[thick,dashed,->] (subset) -- node[right]{\footnotesize $\;\Sbf \subseteq
\text{BK}$} (composition_procedure);
  % Note: If you have no branches, the 2nd pass is not needed
\end{tikzpicture}\vspace{2mm}
\caption{} 
\end{subfigure}
\addtocounter{subfigure}{2}
\begin{subfigure}[b]{0.49\textwidth}\centering
\begin{tikzpicture}[grow=right, sloped]
\node[operator,yshift=5.2cm] {\small $+$}
    child {
        node[operator] {\small $+$}        
            child {
               node[operator] {\small $\times$}        
        child {
                node[operator, label=right:
                    {$\cdots$}] {}
                edge from parent
                node[above] {}
                node[below]  {}
            }
            child {
                node[end, label=right:
                    {LIN$_{\theta^4}$}] {}
                edge from parent
                node[above] {}
                node[below]  {}
            }
        edge from parent         
            node[above] {}
            node[below]  {}
            }
            child {
                node[end, label=right:
                    {WN$_{\theta^3}$}] {}
                edge from parent
                node[above] {}
                node[below]  {}
            }
            edge from parent 
            node[above] {}
            node[below]  {}
    }
    child {
        node[operator] {\small $\times$}        
        child {
                node[end, label=right:
                    {SE$_{\theta^2}$}] {}
                edge from parent
                node[above] {}
                node[below]  {}
            }
            child {
                node[end, label=right:
                    {SE$_{\theta^1}$}] {}
                edge from parent
                node[above] {}
                node[below]  {}
            }
        edge from parent         
            node[above] {}
            node[below]  {}
    };


\node[xshift=-1.4cm,yshift=5.2cm] (K) {$\Parse(\ktheta)=$}; 
\node[draw,rectangle,font=\tiny,text width = 6.7 cm, yshift=0.8cm] (simplification) {
SE $\times$ SE $\;\;\;\;\;\;\;\;\;\;\;\;\;\;\;\;\;\;\;\;\;\;\;\;\;\;\;\;\;\;\,\rightarrow$ SE\\
$\{$SE, PER, C, WN$\} \times$ WN $\;\;\;\;\,\rightarrow$ WN\\
$\{$SE, PER, C, WN, LIN$\} \times$ C $\rightarrow$ $\{$SE, PER, C, WN, LIN$\}$\\
LIN $+$ LIN $\;\;\;\;\;\;\;\;\;\;\;\;\;\;\;\;\;\;\;\;\;\;\;\;\;\;\;\rightarrow$ LIN\\
PER $+$ LIN $ \;\;\;\;\;\;\;\;\;\;\;\;\;\;\;\;\;\;\;\;\;\;\;\;\;\,\rightarrow$ LIN $+$ PER, $\cdots$\\
PER $\times$ LIN $\;\;\;\;\;\;\;\;\;\;\;\;\;\;\;\;\;\;\;\;\;\;\;\;\;\,\rightarrow$ LIN $\times$ PER,$\cdots$\\
};
\node[above=0.0cm of simplification](simplification_with){\small Simplificaton 
 with $\Simplify()$};

\node[yshift=-3cm] (Keq)
{$\text{Struct}(k)=\text{SE} +\text{WN} + \text{LIN} \times ( \cdots )$};
\node[yshift=-0.3cm](helper){};
\node[yshift=-2.7cm](helper2){};
\node[above=-0.1cm of
Keq](interpretation){\small$\kper \rightarrow \text{PER},\;\klin \rightarrow
\text{LIN},\;\kse \rightarrow \text{SE},\cdots$};
\node[above=0.0cm of interpretation](interpretation_with){\small Interpretation 
 with Struct()};
\node[draw, rectangle, below=0.0cm of interpretation_with,minimum width = 8cm, minimum height
=1.2cm] (structbox){};
\node[below=0.3cm of simplification,xshift=-0.2cm]{(c)};
\node[above=0.6cm of simplification_with,xshift=-0.2cm]{(b)};
\end{tikzpicture}
\caption{} 
\end{subfigure}

\begin{subfigure}[b]{0.99\textwidth}\centering
\begin{tabular}{cccc}
\multicolumn{4}{c}{{\bf Base Kernels} (BK)} \rule{0pt}{3ex} \\ 
\small LIN: Linearity &\small PER: Periodicity &\small SE: Smoothness &\small WN: White Noise \rule{0pt}{2ex} \\
\includegraphics[height=2cm]{figs/kernel/kernelLIN.png} & \includegraphics[height=2cm]{figs/kernel/kernelPER.png} & \includegraphics[height=2cm]{figs/kernel/kernelSE.png} & \includegraphics[height=2cm]{figs/kernel/kernelWN.png}\\
\end{tabular}
\begin{tabular}{cccc}
\multicolumn{4}{c}{\bf Composite Structure} \rule{0pt}{0ex}  \\ 
\small LIN + PER: &\small LIN $\times$ PER: &\small SE $\times$ PER: &\small LIN $\times$ LIN: \rule{0pt}{2ex} \\
\small Periodicity with Trend &\small Growing Amplitude &\small Local Periodicity&\small Quadratic \rule{0pt}{2ex} \\
\includegraphics[height=2cm]{figs/kernel/kernelLINplusPER.png} & \includegraphics[height=2cm]{figs/kernel/kernelLINtimesPER.png} & \includegraphics[height=2cm]{figs/kernel/kernelSEplusPER.png}& \includegraphics[height=2cm]{figs/kernel/kernelLINtimesLIN.png}\\
\end{tabular}
\caption{}
\end{subfigure}

\caption{\small(a) Bayesian GP structure learning. A set of
base kernels (BK) with priors on there hyper-parameters serve as hypothesis space
and is supplied as input to the stochastic grammar. The stochastic grammar has
two parts: (i) a sampler that selects a random set $\Sbf$  of primitive kernels from BK
and (ii) a kernel composer that combines the individual base kernels and generates
a composite kernel function
$k_{\thetabf}$. This serves as input for
\gpmem.  We observe value pairs $\xbf,\ybf$ of unstructured time series data on
the bottom of the schematic. (b) An example of composite kernel structure. We
use the Struct($k_{\thetabf}$) to compute a symbolic interpretation for
$k_{\thetabf}$. We use addition and multiplication to combine base kernels.
Base kernels and compositional kernels are shown in (c) alongside ther
interpretation with Struct().}\label{fig:schema}
\end{figure}
%%%%%%%%%%%%%%%%%%%%%%%%%%%%%%%%%%%%%%%%%%%%%%%%%%%%%%%%%%%%
This allows us to read an unstructured time series and automatically output a high-level,
qualitative description of it. The stochastic grammar takes a set of primitive base kernels 
    $\text{BK}=\{k_{\bm{\theta}^1}^1,\cdots,k_{\bm{\theta}^m}^m\}$
    of size $m$, including their corresponding parametrization
    $\thetabf^*=\{\thetabf^1,\cdots,\thetabf^m\}$ (Fig. \ref{fig:schema} (a) and (b))
We depict the input for the
stochastic grammar in Listing \ref{alg:base_kernels}.
\begin{mdframed}
\begin{minipage}{\linewidth}
\small
\belowcaptionskip=-10pt
\begin{lstlisting}[mathescape,label=alg:base_kernels,basicstyle=\selectfont\ttfamily,numbers=none,caption={Initialize
Base Kernels BK and $P(n)$},escapechar=\#]
#\linenumber{1}# // Initialize hyper-parameters
#\linenumber{2}#assume theta_1 = tag(scope="hyper-parameters", block=1, gamma(5,1));
#\linenumber{3}#assume theta_2 = tag(scope="hyper-parameters", block=2, gamma(5,1));
#\linenumber{4}#assume theta_3 = tag(scope="hyper-parameters", block=3, gamma(5,1));
#\linenumber{5}#assume theta_4 = tag(scope="hyper-parameters", block=4, gamma(5,1));
#\linenumber{6}#assume theta_5 = tag(scope="hyper-parameters", block=5, gamma(5,1));
#\linenumber{7}#assume theta_6 = tag(scope="hyper-parameters", block=6, gamma(5,1));
#\linenumber{8}#assume theta_7 = tag(scope="hyper-parameters", block=7, gamma(5,1));
#\linenumber{9}#
#\linenumber{11}# // Make kernels
#\linenumber{12}#assume lin = apply_function(make_linear, theta_1);
#\linenumber{13}#assume per = apply_function(make_periodic, theta_2, theta_3, theta_4);
#\linenumber{14}#assume se  = apply_function(make_squaredexp, theta_5, theta_6);
#\linenumber{15}#assume wn  = apply_function(make_noise, theta_7);
#\linenumber{16}#
#\linenumber{17}#// Initialize the set of primitive base kernels BK 
#\linenumber{18}#assume BK = list(lin, per, se, wn);
\end{lstlisting}
\end{minipage}
\end{mdframed}


We sample a random subset S of
the set of supplied base kernels. S is of size $n \leq m$. We write
%%%%%%%%%%%%%%%%%%%%%%%%%%%%%%%%%%%%%%%%%%%%%%%%%%%%%%%%%%%%
\[
\Sbf = \{K_{\bm{\theta}^i}^i,\cdots,K_{\bm{\theta}^n}^n\}
\sim P(\Sbf = \{K_{\bm{\theta}^i}^i,\cdots,K_{\bm{\theta}^n}^n\} \mid
\text{BK}) 
\]
%%%%%%%%%%%%%%%%%%%%%%%%%%%%%%%%%%%%%%%%%%%%%%%%%%%%%%%%%%%%
with
%%%%%%%%%%%%%%%%%%%%%%%%%%%%%%%%%%%%%%%%%%%%%%%%%%%%%%%%%%%%
\[
P(\Sbf = \{K_{\bm{\theta}^i}^i,\cdots,K_{\bm{\theta}^n}^n\}\mid \text{BK}) =
\frac{n!}{ \mid \Sbf = \{K_{\bm{\theta}^i}^i,\cdots,K_{\bm{\theta}^n}^n\}\mid !}.
\]
%%%%%%%%%%%%%%%%%%%%%%%%%%%%%%%%%%%%%%%%%%%%%%%%%%%%%%%%%%%%

BK is assumed to be fixed as the most general margin of our hypothesis space.
In the following, we will drop it in the notation.
The only building block that we are now missing is how to combine the sampled
base kernels into a compositional covariance function (see Fig. \ref{fig:schema}
(b)). For each interaction $i$, we
have to infer whether the data supports a local interaction or a global interaction,
chosing between one out of two algebraic operators
$\bm{\Omega}_i=\{+,\times\}$. The probability for all such decisions is given by a binomial distribution: 
\begin{equation}
P(\bm{\Omega} \mid \Sbf)= {n \choose r}  p_{+\times}^r (1 - p_{+\times})^{n-r}.
\end{equation}
We can write the marginal probability of a kernel function as 
%%%%%%%%%%%%%%%%%%%%%%%%%%%%%%%%%%%%%%%%%%%%%%%%%%%%%%%%%%%%
\begin{equation}
P(\Krv \mid \xbf,\ybf,\thetabf ) = \iint \limits_{\bm{\Omega},\Sbf}
P(\Krv \mid \xbf,\ybf,\thetabf,\bm{\Omega},\Sbf) \times P(\bm{\Omega} \mid \Sbf)\times
P(\Sbf)\; \text{\bf d}\bm{\Omega}\, \text{\bf d}\Sbf\,
\end{equation}
%%%%%%%%%%%%%%%%%%%%%%%%%%%%%%%%%%%%%%%%%%%%%%%%%%%%%%%%%%%%
with $\bm{\theta}\subseteq \bm{\theta}^*$ as implied by $S$.
For structure learning with \ac{GP} kernels, a composite kernel is
sampled from $P(\Krv)$ and fed into \gpmem. 
The emulator generated by \gpmem\ observes unstructured time series data.
Venture code for the probabilistic grammar is shown in Listing
\ref{alg:grammar}, code for inference with \gpmem\ in Listing
\ref{alg:structureVent}. 


\begin{mdframed}
\begin{minipage}{\linewidth}
\small
\belowcaptionskip=-10pt
\begin{lstlisting}[mathescape,label=alg:grammar,basicstyle=\selectfont\ttfamily,numbers=none,caption={
Stochastic Grammar},escapechar=\#]
#\linenumber{1}#// Select a random subset of the set possible primitive kernels (BK)
#\linenumber{2}#assume primitive_kernel_selection = tag("grammar", 0,
#\linenumber{3}#				 select_primitive_kernels(BK));
#\linenumber{4}#// Construct kernel composition with a composer procedure
#\linenumber{5}#assume kernel_composer = proc(l) {
#\linenumber{6}#  if (size(l) <= 1) {
#\linenumber{7}#    first(l)
#\linenumber{8}#  } else {
#\linenumber{9}#       if (bernoulli()) {
#\linenumber{10}#            add_funcs(first(l),  kernel_composer(rest(l)))
#\linenumber{11}#       } else {
#\linenumber{12}#            mult_funcs(first(l), kernel_composer(rest(l)))
#\linenumber{13}#    }
#\linenumber{14}#  }
#\linenumber{15}#};
#\linenumber{16}#
#\linenumber{17}#assume K = tag("grammar", 1,
	        	kernel_composer(primitive_kernel_selection));
\end{lstlisting}

\end{minipage}
\end{mdframed}

\begin{mdframed}
\begin{minipage}{\linewidth}
\small
\belowcaptionskip=-10pt
\begin{lstlisting}[mathescape,label=alg:structureVent,basicstyle=\selectfont\ttfamily,numbers=none,caption={\gpmem\
inference for structure
learning: },escapechar=\#]
#\linenumber{1}#// Apply gpmem 
#\linenumber{2}#assume (f_compute, f_emu) = gpmem(f_look_up, K);
#\linenumber{3}#// Probe all data points
#\linenumber{4}#for (n = 0; n < size(data); n++) { 
#\linenumber{5}#	predict f_compute(first(lookup(data, n)))};
#\linenumber{6}#// Perform inference
#\linenumber{7}#infer repeat(200, do(
#\linenumber{8}#	mh(scope="grammar", steps=1),
#\linenumber{9}#	mh(scope="hyper-parameters", steps=2)));
\end{lstlisting}

\end{minipage}
\end{mdframed}






Many equivalent covariance structures can be sampled due to covariance function algebra
and equivalent representations with different parameterization~\citep{lloyd2014automatic}.
To inspect the posterior of these equivalent structures we convert each kernel expression
into a sum of products and subsequently simplify. All base kernels can be found in Appendix A,
rules for this simplification can be found in Appendix B.
Kernel structure comes with symbolic interpretations. To read kernel function $k$
and apply the simplifications described above, we apply, Struct$(k)$. For
examle, we write
\[
\Struct(\klin)=\text{LIN},
\]
which translates a function into a symbolic expression, see Appendix C. 
%%%%%%%%%%%%%%%%%%%%%%%%%%%%%%%%%%%%%%%%%%%%%%%%%%%%%%%%%%%%%%%%%%%%%%%%%%%%%%%%%
%%%%%%%%%%%%%%%%%%%%%%%%%     Mauna result      %%%%%%%%%%%%%%%%%%%%%%%%%%%%%%%%%%
%%%%%%%%%%%%%%%%%%%%%%%%%%%%%%%%%%%%%%%%%%%%%%%%%%%%%%%%%%%%%%%%%%%%%%%%%%%%%%%%%
We defined a simple space of covariance structures in a way that allows us to produce results coherent with 
work presented in Automatic Statistician. The results are illustrated with two data sets.

\myparagraph{Mauna Loa  CO$_2$ data}
%%%%%%%%%%%%%%%%%%%%%%%%%%%%%%%%%%%%%%%%%%%%%%%%%%%%%%%%%%%%%%%%%%%%%%%%%%%%%%%%%
\begin{figure}
\centering
 \addtolength\abovedisplayskip{-1\baselineskip}%
  \addtolength\belowdisplayskip{-1\baselineskip}%
  
\begin{tikzpicture}
\node[] (data) {\includegraphics[width=0.7\textwidth]{figs/mauna_data.png}};
\node[below = 2cm of data] (post_param_helper) {};
\node[above = 1cm of post_param_helper] (post_param_helper_1) {};
\node[below = 1cm of post_param_helper] (post_param_helper_2) {};
\node[left = -2cm of post_param_helper] (post_param) {\includegraphics[width=0.6\textwidth]{figs/mauna_sample_1.png}};
\node[draw,rectangle,color=red,dashed,right = 2cm of post_param_helper,yshift=0.5cm] (zoom) {\includegraphics[width=0.2\textwidth]{figs/mauna_zoom.png}};
\node[draw,rectangle,color=red,dashed,left = 2.8cm of zoom, minimum width = 1.5cm, minimum height = 1.2cm] (zoom_in) {};
\node[below = 0.7cm of post_param_helper_2] (posterior) {\includegraphics[width=0.6\textwidth]{figs/mauna_structure.png}};

\node[draw,rectangle,below = 0.7cm of posterior] (formula) {\color{black}
$\mathbf{K}=\text{LIN} + \text{PER} + \text{SE} + \text{WN}$};
\node[draw,rectangle,below = 0.7cm of formula] (formula_param_1) {\color{black}\small
$= 2.7^2(x x^\prime) + 5.6^2 \exp \bigg( \frac{2 \sin^2 ( \pi (x - x^\prime)/3.7}{6.4^2} \bigg)
+ 0.4^2 \exp(-\frac{(x-x^\prime)^2}{2 \times 6.3^2}) +  1.9^2 \delta_{x,x^\prime} \label{eq:WN}$ };


\node[draw,rectangle,below = 0.7cm of formula_param_1,text width = \textwidth,minimum height = 1.5cm] (paragraph){\color{black}\footnotesize {\bf Qualitative Interpretation}:
The posterior peaks at a kernel structure with four additive components. Additive components hold globally, that is there are no higher level, qualitative aspects of the data that vary with the input space. The additive components are as follows: (i) a linearly increasing function or trend; (ii) a periodic function; (iii) a smooth function; and (iv) white noise.};







\node[draw, rectangle, left = -1.75cm of posterior,minimum width = 0.45cm, minimum height = 5.8cm,yshift=0.2cm] (mark_structure) {};
%\node[draw,very thick, rectangle, below = 1.1cm of data,minimum width = \textwidth, minimum height = 15cm] (posterior_frame) {};

%\node[left = 1.3cm of mark_structure] (paragraph_helper){};
\node[below =0.4cm of mark_structure,inner sep = 0pt,outer sep=0pt] (formula_helper) {};
\node[above =0.7cm of formula,inner sep = 0pt,outer sep=0pt] (formula_helper_2) {};

%\draw[-,dashed] (mark_structure.south) -- (formula_helper);
%\draw[-,dashed] (formula_helper) -- (formula_helper_2);
%\draw[->,dashed] (mark_structure) -- (paragraph_helper);
%\draw[->,dashed] (formula_helper_2) -- (formula);

\draw[->] (data) -- (post_param_helper_1);
\draw[->] (post_param_helper_2) -- (posterior);
\draw[->] (formula_helper_2) -- (formula);
\draw[-] (mark_structure) -- (formula_helper);
\draw[-] (formula_helper) -- (formula_helper_2);
\draw[->] (formula) -- (formula_param_1);
\draw[->] (formula_param_1) -- (paragraph);

\draw[->,dashed,red] (zoom_in) -- (zoom);

%\draw[->,line width=1pt,double distance=2pt] (data) -- (post_param);
\end{tikzpicture}
\addtolength\abovedisplayskip{1\baselineskip}%
\addtolength\belowdisplayskip{1\baselineskip}%



\caption{\small Structure Learning. Starting with raw data (a), we fit a \ac{GP}
(b) and compute the posterior distribution on structures (c). We take a sample
of the peak of this distribution ($\text{LIN}+\text{PER}+\text{SE}+\text{WN}$)
including its parameters and write it in functional form (d). We depict the
human readable interpretation (e). We used (d) to plot (b).}\label{fig:posterior}
\end{figure}
%%%%%%%%%%%%%%%%%%%%%%%%%%%%%%%%%%%%%%%%%%%%%%%%%%%%%%%%%%%%%%%%%%%%%%%%%%%%%%%%%
We illustrate results in Fig \ref{fig:posterior}. In Fig \ref{fig:posterior} (a) we depict the raw data. 
We see mean centered CO$_2$ measurements of the Mauna Loa Observatory, an atmospheric
baseline station on Mauna Loa, on the island of Hawaii. 
A description of the data set  can be found in  \citealp[][chapter 5]{rasmussen2006gaussian}.  
We use those raw data to compute a posterior on structure, parameters and \ac{GP}
samples.
The latter are shown in  Fig \ref{fig:posterior} (b)
where we zoom in to show how the posterior captures the error bars
adequately.
This posterior of the \ac{GP} is generated with a random sample from the parameters
of the peak of the distribution on structure (Fig \ref{fig:posterior} (c)).
We differentiate between a posterior distribution on kernel functions and on
distribution on symbolic expressions describing different kernel structures. 
This allows us to compute the posterior of symbollically equivalent structures,
such as $\Struct(\klin + \kper)=\Struct(\kper + \klin)$. Both structures yield and addition of a linear kernel and a periodic kernel, that is LIN + PER.
We coin the random variable over symbolic expressions for kernels as $\Ksrv$.
The distribution peaks at:
%%%%%%%%%%%%%%%%%%%%%%%%%%%%%%%%%%%%%%%%%%%%%%%%%%%%%%%%%%%%%%%%%%%%%%%%%%%%%%%%%
\begin{equation}
\Ksrv=\text{LIN} + \text{PER} + \text{SE} + \text{WN}.
\end{equation}
%%%%%%%%%%%%%%%%%%%%%%%%%%%%%%%%%%%%%%%%%%%%%%%%%%%%%%%%%%%%%%%%%%%%%%%%%%%%%%%%%
We write this kernel equation out in Fig \ref{fig:posterior} (d).
This kernel structure has a natural language interpretation that we spell out in
Fig \ref{fig:posterior} (e), explaining that 
the posterior peaks at a kernel structure with four additive components.
Each of which holds globally, that is there are no higher level, qualitative aspects
of the data that vary with the input space. The additive components for this result are as follows:
\begin{itemize}
\item a linearly increasing function or trend; 
\item a periodic function;
\item a smooth function; and
\item white noise.
\end{itemize}
 



Previous work on automated kernel discovery~\citep{duvenaud2013structure} illustrated the Mauna Loa data using an RQ kernel.
We resort to the white noise kernel instead of RQ (similar to \citep{lloyd2014automatic}).


%%%%%%%%%%%%%%%%%%%%%%%%%%%%%%%%%%%%%%%%%%%%%%%%%%%%%%%%%%%%%%%%%%%%%%%%%%%%%%%%%
%%%%%%%%%%%%%%%%%%%%%%%%%     Airline result   %%%%%%%%%%%%%%%%%%%%%%%%%%%%%%%%%%
%%%%%%%%%%%%%%%%%%%%%%%%%%%%%%%%%%%%%%%%%%%%%%%%%%%%%%%%%%%%%%%%%%%%%%%%%%%%%%%%
\myparagraph{Airline Data}
The second data set (Fig. \ref{fig:posterior_airline}) we depict results for is  the airline 
data set describing monthly totals of international airline passengers (\citealp{box2011time}, according to \citealp{duvenaud2013structure}). 
%%%%%%%%%%%%%%%%%%%%%%%%%%%%%%%%%%%%%%%%%%%%%%%%%%%%%%%%%%%%%%%%%%%%%%%%%%%%%%%%
\begin{figure}
\centering
 \addtolength\abovedisplayskip{-1\baselineskip}%
  \addtolength\belowdisplayskip{-1\baselineskip}%
  
\begin{tikzpicture}
\node (datatitle) {\small Raw Data};
\node[below = -0.25cm of datatitle](data) {\includegraphics[width=0.65\textwidth]{figs/airline_data.png}};
\node[below= 1.2cm of data] (post_param) {\includegraphics[width=0.65\textwidth]{figs/airline_sample_28.png}};


\node[below = 1.2cm of post_param] (posterior) {\includegraphics[width=0.55\textwidth]{figs/airline_structure.png}};


\node[draw,rectangle,below = 1.2cm of posterior] (formula_param_1) {\small\color{black}
$= 7.47^2(x x^\prime) +
\Bigg(0.27^2 \exp(-\frac{(x-x^\prime)^2}{2 \times 4.63^2}) \times 
7.34^2 \exp \bigg( \frac{2 \sin^2 ( \pi (x - x^\prime)/4.4}{4.55^2} \bigg)\Bigg)
+ 2.93^2 \delta_{x,x^\prime} \label{eq:WN}$ };


\node[draw,rectangle,below = 1.2cm of formula_param_1,text width =0.9\textwidth,minimum height = 1.5cm, font=\footnotesize] (paragraph){ 
The posterior peaks at a kernel structure with three additive components. Additive components hold globally, that is there are no higher level, qualitative aspects of the data that vary with the input space. The additive components are as follows: (i) a linearly increasing function or trend; (ii) a approximate periodic function; and (iv) white noise.};







\node[draw, rectangle, left = -1.7cm of posterior,minimum width = 0.45cm, minimum height = 5.2cm,yshift=0.2cm] (mark_structure) {};
%\node[draw,very thick, rectangle, below = 1.1cm of data,minimum width = \textwidth, minimum height = 15cm] (posterior_frame) {};

%\node[left = 1.3cm of mark_structure] (paragraph_helper){};
\node[below =0.45cm of mark_structure,inner sep = 0pt,outer sep=0pt] (formula_helper) {};
\node[above =1.2cm of formula_param_1,inner sep = 0pt,outer sep=0pt] (formula_helper_2) {};

%\draw[-,dashed] (mark_structure.south) -- (formula_helper);
%\draw[-,dashed] (formula_helper) -- (formula_helper_2);
%\draw[->,dashed] (mark_structure) -- (paragraph_helper);
%\draw[->,dashed] (formula_helper_2) -- (formula);

\draw[->] (data) -- node[right]{\small $\hat \fbf \sim
\mathcal{N}(\hat{\bm{\mu}},\hat\Kbf)$} (post_param_helper_1);
\draw[->] (post_param_helper_2) -- node[right]{\small Marginal Structure} (posterior);
\draw[->] (formula_helper_2) -- node[right] {\small
$\bm{\theta}=\{7.47,0.27,4.63,7.34,4.4,4.55,2.93\}$} (formula_param_1);
\draw[-] (mark_structure) -- node[left, yshift=-0.3cm] {\small $\Ktheta$} (formula_helper);
\draw[-] (formula_helper) --(formula_helper_2);
\draw[->] (formula_param_1) -- node[right]{\small Qualitative Interpretation} (paragraph);


\node[left=0.3cm of paragraph] (e){(e)}; 

\node[above=1.9cm of e] (d) {(d)}; 
\node[above=5.0cm of d] (c) {(c)}; 
\node[above=4.7cm of c] (b) {(b)}; 
\node[above=4.0cm of b] (a) {(a)}; 
\end{tikzpicture}
\addtolength\abovedisplayskip{1\baselineskip}%
\addtolength\belowdisplayskip{1\baselineskip}%



\caption{\small Structure Learning. Starting with raw data (a), we fit a \ac{GP}
(b) and compute the posterior distribution on structures (c). We take a sample
of the peak of this distribution ($\text{LIN}+\text{PER} \times \text{SE}+\text{WN}$)
including its parameters and write it in functional form (d). We depict the
human readable interpretation (e). We used (d) to plot (b).}\label{fig:posterior_airline}
\end{figure}
%%%%%%%%%%%%%%%%%%%%%%%%%%%%%%%%%%%%%%%%%%%%%%%%%%%%%%%%%%%%%%%%%%%%%%%%%%%%%%%%

We illustrate results for this data set in Fig \ref{fig:posterior_airline}. In Fig \ref{fig:posterior_airline} (a) we depict the raw data. 
Again, the data is mean centered and we use it to 
compute a posterior on structure, parameters and \ac{GP}
samples.
The latter are shown in  Fig \ref{fig:posterior_airline} (b).
This posterior of the \ac{GP} is generated with a random sample from the parameters
of the peak of the distribution on structure (Fig \ref{fig:posterior_airline} (c)).
The posterior over symbolic kernel expressions peaks at:
%%%%%%%%%%%%%%%%%%%%%%%%%%%%%%%%%%%%%%%%%%%%%%%%%%%%%%%%%%%%%%%%%%%%%%%%%%%%%%%%%
\begin{equation}
\Ksrv=\text{LIN} +  \text{SE} \times \text{PER}+ \text{WN}.
\end{equation}
%%%%%%%%%%%%%%%%%%%%%%%%%%%%%%%%%%%%%%%%%%%%%%%%%%%%%%%%%%%%%%%%%%%%%%%%%%%%%%%%%
We write this Kernel equation out in Fig \ref{fig:posterior_airline} (d).
This kernel structure has a natural language interpretation that we spell out in
Fig \ref{fig:posterior_airline} (e), explaining that 
the posterior peaks at a kernel structure with three additive components.
Additive components hold globally, that is there are no higher level, qualitative aspects
of the data that vary with the input space.
The additive components are as follows: 
\begin{itemize}
\item a linearly increasing function or trend;
\item a approximate periodic function; and
\item  white noise.
\end{itemize}
Both datasets served as illustrations in the Automatic Statistician project.



%%%%%%%%%%%%%%%%%%%%%%%%%%%%%%%%%%%%%%%%%%%%%%%%%%%%%%%%%%%%%%%%%%%%%%%%%%%%%%%%%
%%%%%%%%%%%%%%%%%%%%%%%%% Queries for time series %%%%%%%%%%%%%%%%%%%%%%%%%%%%%%%
%%%%%%%%%%%%%%%%%%%%%%%%%%%%%%%%%%%%%%%%%%%%%%%%%%%%%%%%%%%%%%%%%%%%%%%%%%%%%%%%%
%%%%%%%%%%%%%%%%%%%%%%%%%%%%%%%%%%%%%%%%%%%%%%%%%%%%%%%%%%%%%%%%%%%%%%%%%%%%%%%%%
\myparagraph{Querying time series}
With our Bayesian approach to structure learning we can gain valuable insights
into time series data that were previously unavailable.
This is due to our ability to estimate posterior marginal probabilities over the kernel structure.
Over this marginal, we define boolean search operations that allow us to query the data
for the probability of certain structures to hold true globally.
%%%%%%%%%%%%%%%%%%%%%%%%%%%%%%%%%%%%%%%%%%%%%%%%%%%%%%%%%%%%%%%%%%%%%%%%%%%%%%%%%
\begin{equation}
\label{eq:bool_present}
P(\Ksrv \mid \xbf,\ybf,\thetabf) = \frac{1}{T}
\sum\limits_{t=1}^T f(\Ksrv^t)\;\;\text{where}\, f(\Ksrv^t) = \begin{cases}
  1, & \text{if } \Ksrv \underset{global}{\in} \Ksrv^t, \\
  0, & \text{otherwise}.
\end{cases} 
\end{equation}
%%%%%%%%%%%%%%%%%%%%%%%%%%%%%%%%%%%%%%%%%%%%%%%%%%%%%%%%%%%%%%%%%%%%%%%%%%%%%%%%%
to ask whether it is true that a global structure $\Ksrv$ is present. $T$
is the number of all posterior samples for $\Ksrv$ and $\Ksrv^t$ is one such
sample. 
We can now ask simple questions, for example:
\begin{quotation}
Is there white noise in the data?
\end{quotation}
where we set $\Ksrv = $WN in (\ref{eq:bool_present}).
We can also formulate more sophisticated search operations using Boolean operators such as AND ($\land$) and OR ($\lor$).
The AND operator is defined as follows:
%%%%%%%%%%%%%%%%%%%%%%%%%%%%%%%%%%%%%%%%%%%%%%%%%%%%%%%%%%%%%%%%%%%%%%%%%%%%%%%%%
\[
P(\Ksrv^a \land \Ksrv^b \mid \xbf,\ybf, \thetabf)  = \frac{1}{N}
\sum\limits_{n=1}^N f(\Ksrv^t)\;\;
\]
%%%%%%%%%%%%%%%%%%%%%%%%%%%%%%%%%%%%%%%%%%%%%%%%%%%%%%%%%%%%%%%%%%%%%%%%%%%%%%%%%
where
\[
f(\Ksrv^t) = \begin{cases}
  1, & \text{if } \Ksrv^a\, \text{and}\, \Ksrv^b  \underset{global}{\in} \Ksrv^t, \\
  0, & \text{otherwise}\end{cases}.
\]
By estimating $P(\text{LIN} \land \text{WN} \mid \xbf, \ybf, \thetabf)$ we can use this operator to ask questions such as 
\begin{quotation}
Is there a Linear component AND a white noise in the data? 
\end{quotation}
Finally, we define the logical OR as
%%%%%%%%%%%%%%%%%%%%%%%%%%%%%%%%%%%%%%%%%%%%%%%%%%%%%%%%%%%%%%%%%%%%%%%%%%%%%%%%%
\begin{align*}
P(\Ksrv^a \lor \Ksrv^b \mid \xbf, \ybf, \thetabf)
=& P(\Ksrv^a \mid \xbf, \ybf, \thetabf) + P(\Ksrv^b \mid \xbf, \ybf, \thetabf)\\
 &- P(\Ksrv^a \land \Ksrv^b \mid \xbf, \ybf, \thetabf)
\end{align*}
%%%%%%%%%%%%%%%%%%%%%%%%%%%%%%%%%%%%%%%%%%%%%%%%%%%%%%%%%%%%%%%%%%%%%%%%%%%%%%%%%
which allows us to ask questions about structures that are logically connected with OR, such as:
\begin{quotation}
Is there white noise or heteroskedastic noise?
\end{quotation}
by estimating $P(\text{LIN} \times \text{WN}\;\;{\large\lor}\;\; \text{WN} \mid
\xbf, \ybf, \thetabf)$.
We know that noise can either be heteroskedastic or white,
and we also know due to simple manipulations using kernel algebra
that  $\text{LIN} \times \text{WN}$ and $\text{WN}$ are the only possible ways to construct noise with kernel composition, we see that we can generalize the 
question above to:
\begin{quotation}
Is there noise in the data? 
\end{quotation}
where we write the marginal posterior on qualitative structure for noise:
%%%%%%%%%%%%%%%%%%%%%%%%%%%%%%%%%%%%%%%%%%%%%%%%%%%%%%%%%%%%%%%%%%%%%%%%%%%%%%%%%
\begin{equation}
P(\Ksrv^{\text{noise}} \mid \xbf, \ybf, \thetabf) = P(\text{LIN} \times
\text{WN}\;\;{\large\lor}\;\; \text{WN} \mid \xbf, \ybf, \thetabf).
\end{equation}
%%%%%%%%%%%%%%%%%%%%%%%%%%%%%%%%%%%%%%%%%%%%%%%%%%%%%%%%%%%%%%%%%%%%%%%%%%%%%%%%%
Note that this allows us to start with general queries and 
subsequently formulate follow up queries that go into more detail.
For example, we could start with a general query, such as:
\begin{quotation}
What is the probability of a trend, a recurring pattern {\bf and} noise in the data?
\end{quotation}
and then follow up with more detailed questions (Fig \ref{fig:query}).
%%%%%%%%%%%%%%%%%%%%%%%%%%%%%%%%%%%%%%%%%%%%%%%%%%%%%%%%%%%%%%%%%%%%%%%%%%%%%%%%%
\begin{figure}
\centering
\footnotesize
% \includegraphics[width=.153\textwidth]{figs/gpSamples/main.png}
\begin{tabular}{cccccc}
\multicolumn{6}{c}{ \includegraphics[width=0.6\textwidth]{figs/mauna_data.png}}     \\                                               
\multicolumn{6}{c}{\tikzmark{a}}     \\  
\multicolumn{6}{c}{}     \\   
\multicolumn{6}{c}{\tikzmark{b}}     \\  
\multicolumn{6}{c}{ \includegraphics[width=0.5\textwidth]{figs/mauna_structure.png}}     \\                                               
\multicolumn{6}{c}{\tikzmark{c}}     \\  
\multicolumn{6}{c}{}     \\   
\multicolumn{6}{c}{\tikzmark{d}}     \\           
\multicolumn{6}{c}{\normalsize \color{blue} What is the probability of a trend, a recurring pattern {\bf and} noise in the data?}     \\               
\multicolumn{6}{c}{$P\big((\text{LIN}\lor\text{LIN}\times\text{SE})\land
(\text{PER}\lor\text{PER}\times\text{SE}\lor\text{PER}\times\text{LIN})\land
(\text{WN}\lor\text{LIN}\times\text{WN})\big) = 0.36
$}                                                     \\
         \multicolumn{2}{c}{ \tikzmark{trend_part}}     &  \multicolumn{2}{c}{$\;\;\;\;$\tikzmark{recurring_part}}   &     \multicolumn{2}{c}{\tikzmark{noise_part}}                               \\
          &          &  &  &  &                            \\
\multicolumn{2}{c}{\tikzmark{trend}}& & &   \multicolumn{2}{c}{ \tikzmark{noise}} \\  
\multicolumn{2}{c}{\normalsize \color{blue} Is there a trend?}& & &   \multicolumn{2}{c}{ \normalsize \color{blue} Is there noise?} \\
\multicolumn{2}{c}{$P(\text{LIN}\lor\text{LIN}\times\text{SE}) = 0.65$}& & &   \multicolumn{2}{c}{$P(\text{WN}\lor\text{LIN}\times\text{WN}) = 0.75$} \\
\multicolumn{2}{c}{\includegraphics[width=.15\textwidth]{figs/gpSamples/trend.png}}& & &   \multicolumn{2}{c}{\includegraphics[width=.15\textwidth]{figs/gpSamples/noise.png}} \\
\multicolumn{2}{c}{\tikzmark{trend_below}}& & &   \multicolumn{2}{c}{ \tikzmark{noise_below}} \\  
        &          &  &       &       &                            \\
         \tikzmark{lin}  &\tikzmark{linse}             &         &       &                 \tikzmark{linwn}  &\tikzmark{wn}            \\

           {\normalsize \color{blue} A linear trend?}  &{\normalsize \color{blue} A smooth trend?}   &         &       &                 {\normalsize \color{blue} Heteroskedastic noise?}  &{\normalsize \color{blue} White noise?}           \\
          $P(\text{LIN}) = 0.63$& $P(\text{LIN}\times\text{SE}) = 0.02$         &          &    &$P(\text{LIN}\times\text{WN}) = 0$  &$P(\text{WN}) = 0.75$    \\
                \includegraphics[width=.15\textwidth]{figs/gpSamples/lin.png}& \includegraphics[width=.15\textwidth]{figs/gpSamples/selin.png}    &          &    &\includegraphics[width=.15\textwidth]{figs/gpSamples/linwn.png}  &\includegraphics[width=.15\textwidth]{figs/gpSamples/wn.png}   \\
         &          &         &\tikzmark{recurring}       &                &          \\
      \multicolumn{6}{c}{\normalsize \color{blue} Is there repeating structure?} \\
  \multicolumn{6}{c}{$P(\text{PER}\lor\text{PER}\times\text{SE}\lor\text{PER}\times\text{LIN}) = 0.73$} \\
   \multicolumn{6}{c}{\includegraphics[width=.15\textwidth]{figs/gpSamples/recurring.png}$\;\;\;\;\;\;$} \\
     & &    \multicolumn{2}{c}{\tikzmark{recurring_below}} & &\\
      &          &  &       &       &                            \\
         &  \tikzmark{per}        &   \multicolumn{2}{c}{\tikzmark{perse}}     &   \tikzmark{perlin}    &    \\
        \multicolumn{2}{c}{$P(\text{PER}) = 0.32$}         &           \multicolumn{2}{c}{$P(\text{PER}\times\text{SE}) = 0.34$}         &                   \multicolumn{2}{c}{$P(\text{PER}\times\text{LIN}) = 0.07$}           \\
        \multicolumn{2}{c}{\includegraphics[width=.15\textwidth]{figs/gpSamples/per.png}}         &           \multicolumn{2}{c}{\includegraphics[width=.15\textwidth]{figs/gpSamples/seper.png}}         &                   \multicolumn{2}{c}{\includegraphics[width=.15\textwidth]{figs/gpSamples/perlin.png}}           
        \end{tabular}
 \begin{tikzpicture}[overlay, remember picture, yshift=.25\baselineskip, shorten >=.5pt, shorten <=.5pt]
   \draw [->] ({pic cs:a}) to ({pic cs:b});
      \draw [->] ({pic cs:c}) to ({pic cs:d});
    \draw [->] ({pic cs:recurring_part}) to ({pic cs:recurring});
    
     \draw [->] ({pic cs:recurring_below}) to ({pic cs:per});
      \draw [->] ({pic cs:recurring_below}) to ({pic cs:perse});
     \draw [->] ({pic cs:recurring_below}) to ({pic cs:perlin});
    
    
      \draw [->] ({pic cs:trend_part}) to ({pic cs:trend});
        \draw [->] ({pic cs:trend_below}) to ({pic cs:lin});
    \draw [->] ({pic cs:trend_below}) to ({pic cs:linse});
    
     \draw [->] ({pic cs:noise_part}) to ({pic cs:noise});
       \draw [->] ({pic cs:noise_below}) to ({pic cs:wn});
         \draw [->] ({pic cs:noise_below}) to ({pic cs:linwn});
  \end{tikzpicture}
%  \multicolumn{3}{c}{$P(\text{PER}\lor\text{PER}\times\text{SE}\lor\text{PER}\times\text{LIN})$}
\caption{{\bf Querying structural motifs in in time series using posterior inference
over kernel structure.} The kernel structure serves as a way to formulate
natural language questions about the data (blue). The initial question of interest
(top) is a fairly
general one: "What is the probability of a trend, a recurring
pattern and noise in the data?" Below the natural language version of this
question, the same question is formulated as an inference problem (black) over the
marginal probability on kernels with Boolean operators AND ($\land$) and OR ($\lor$). 
To gain  a deeper understanding of specific motifs in the time series more specific queries can
be written.
On the right, a query asks whether there is noise in the data (blue) by computing the disjunction of the marginal
of a global white noise kernel and a multiplication between a linear and a white
noise kernel (black). Samples from the predictive prior $\ybf_*$ of such kernels give an
indication of the qualitative aspects that a kernel structure implies (coloured curves below
the marginal). 
If the probability that there is noise in the data is high then it makes sense
to drill even deeper asking more detailed questions. With regards to noise, this
translates to querying whether or not the data supports the hypothesis that there is
heteroskedastic noise or white noise. Queries for motifs of repeating structure
are shown in the middle of the tree, queries related to trends on the left.}\label{fig:query}
\end{figure}
%%%%%%%%%%%%%%%%%%%%%%%%%%%%%%%%%%%%%%%%%%%%%%%%%%%%%%%%%%%%%%%%%%%%%%%%%%%%%%%%%
This way of querying your data for their statistical implications is in stark contrast to what previous research in automatic kernel construction was able to provide.
We could view our approach as a time series search engine which allows us to test whether or not certain structures can be found
in an available time series.
Another way to view this approach is as a new language to interact with the world.
Real-world observations often come with time-stamps and in form
of continuous valued sensor measurements.  
We provide the toolbox to query such observations in a similar manner as
one would query a knowledge base in a logic programming language.








\subsection{Bayesian optimization}

\label{sec:thompson}



We introduce Thompson sampling, the basic solution strategy
underlying the Bayesian optimization with \gpmem.
Thompson sampling~\cite{thompson1933likelihood} is a widely used Bayesian
framework for addressing the trade-off between exploration and exploitation in
multi-armed (or continuum-armed) bandit problems.  
We cast the multi-armed bandit problem as a one-state Markov
decision process, and describe how Thompson sampling can be used to choose
actions for that \ac{MDP}.

Here, an agent is to take a sequence of actions $a_1, a_2,
\ldots$ from a (possibly infinite) set of possible actions $\Acal$.  After each
action, a reward $r \in \R$ is received, according to an unknown conditional
distribution $P_\true\pn{r \mvert a}$.  The agent's goal is to maximize the
total reward received for all actions in an online manner.  In Thompson
sampling, the agent accomplishes this by placing a prior distribution
$P(\ttheta)$ on the possible ``contexts'' $\ttheta \in \Theta$.  Here a context
is a believed model of the conditional distributions $\{P\pn{r \mvert a}\}_{a
\in \Acal}$, or at least, a believed statistic of these conditional
distributions which is sufficient for deciding an action $a$.  If actions are
chosen so as to maximize expected reward, then one such sufficient statistic is
the believed conditional mean $V\pn{a \mvert \ttheta} = \Ebkt{r \mvert
a;\ttheta}$, which can be viewed as a believed value function.  For
consistency with what follows, we will assume our context $\ttheta$ takes the
form $(\theta, \abf_\past, \rbf_\past)$ where $\abf_\past$ is the vector of past
actions, $\rbf_\past$ is the vector of their rewards, and $\theta$ (the
``semicontext'') contains any other information that is included in the context.

In this setup, Thompson sampling has the following steps:
\begin{algorithm}[H]
  \singlespacing
  Repeat as long as desired:
  \begin{enumerate}
    \item\label{itm:thompson-step-sample} {\bf Sample.} Sample a semicontext
      $\theta \sim P(\theta)$.
    \item\label{itm:thompson-step-search} {\bf Search (and act).} Choose an
      action $a \in \Acal$ which (approximately) maximizes $V\pn{a \mvert
      \ttheta} = \Ebkt{r \mvert a; \ttheta} = \Ebkt{r \mvert a;\,\theta,
      \abf_\past, \rbf_\past}$.
    \item {\bf Update.} Let $r_\true$ be the reward received for action $a$.
      Update the believed distribution on $\theta$, i.e., $P(\theta) \gets
      P_\rmnew(\theta)$ where $P_\rmnew(\theta) = P\pn{\theta \mvert a \mapsto
      r_\true}$.
  \end{enumerate}
  \caption{Thompson sampling.}
  \label{alg:thompson}
\end{algorithm}
Note that when $\Ebkt{r|a;\ttheta}$ (under the sampled value of $\theta$ for
some points $a$) is far from the true value $\Ebkt[P_\true]{r \mvert a}$, the
chosen action $a$ may be far from optimal, but the information gained by probing
action $a$ will improve the belief $\ttheta$.  This amounts to ``exploration.''
When $\Ebkt{r \mvert a;\ttheta}$ is close to the true value except at points $a$
for which $\Ebkt{r \mvert a;\ttheta}$ is low, exploration will be less likely to
occur, but the chosen actions $a$ will tend to receive high rewards.  This
amounts to ``exploitation.'' The trade-off between exploration and exploitation
is illustrated in Figure \ref{fig:slide2}.  Roughly speaking, exploration will
happen until the context $\ttheta$ is reasonably sure that the unexplored
actions are probably not optimal, at which time the Thompson sampler will
exploit by choosing actions in regions it knows to have high value.

\begin{figure}
  \centering
  \includegraphics[width=\linewidth]{figs/slide2.pdf}
  \caption{
    Two possible actions (in green) for an iteration of Thompson sampling.  The
    believed distribution on the value function $V$ is depicted in red.  In this
    example, the true reward function is deterministic, and is drawn in blue.
    The action on the right receives a high reward, while the action on the left
    receives a low reward but greatly improves the accuracy of the believed
    distribution on $V$.  The transition operators $\tau_\search$ and
    $\tau_\update$ are described in Section \ref{sec:math-spec}.
  }
  \label{fig:slide2}
\end{figure}

Typically, when Thompson sampling is implemented, the search over contexts
$\ttheta \in \Theta$ is limited by the choice of representation.  In
traditional programming environments, $\theta$ often consists of a few
numerical parameters for a family of distributions of a fixed functional
form.  With work, a mixture of a few functional forms is possible; but
without probabilistic programming machinery, implementing a rich context
space $\Theta$ would be an unworkably large technical burden.  In a
probabilstic programing language, however, the representation of
heterogeneously structured or infinite-dimensional context spaces is quite
natural.  Any computable model of the conditional distributions
$\br{P\pn{r \mvert a}}_{a \in \Acal}$ can be represented as a stochastic
procedure $(\lambda (a) \ldots)$.  Thus, for computational Thompson sampling,
the most general context space $\widehat\Theta$ is the space of program texts.
Any other context space $\Theta$ has a natural embedding as a subset of
$\widehat\Theta$.
\begin{comment}
\myparagraph{Thompson Sampling with a Statistical Memoizer}
\label{sec:thompson-mem-em}
Thompson sampling as described above can be expressed compactly in terms of a
statistical memoizer, as in Listing \ref{lst:thompson-mem-em} (see also Figure
\ref{fig:slide1}).  Here and in what follows, to simplify the treatment, we
assume the true reward function is deterministic, i.e., for each fixed $a$, the
conditional distribution $P_\true\pn{r \mvert a}$ is a delta distribution.  We
thus treat the notations $a,r,\abf_\past,\rbf_\past$ as synonyms for
$x,y,\xbf_\past,\ybf_\past$, respectively, differing only in that they carry the
connotations of ``action'' and ``reward.''

\begin{mdframed}
\begin{minipage}{\linewidth}
\small
\belowcaptionskip=-10pt
\begin{lstlisting}[escapechar=\#,language=Venture,label=lst:thompson-mem-em,caption={
  Code template for Thompson sampling in pseudo-Venture using a statistical
  memoizer.  The choice of statistical memoizer (\texttt{mem\&em}), prior on
  semicontexts (\texttt{prior\_on\_semicontexts}), and approximation strategy
  for maximizing the emulator (\texttt{argmax}) are not included.
}]
#\linenumber{1}#assume theta = tag(quote( theta ), 0,  prior_on_semicontexts())
#\linenumber{2}#assume (r_probe r_emulate) = mem&em( do_action, theta)

#\linenumber{3}#infer repeat(15, do(pass,
#\linenumber{4}#     a = mc_argmax(r_emulate, quote(_)),
#\linenumber{5}#     r_probe(a)
#\linenumber{6}#     mh(quote(theta), one, 50)))
\end{lstlisting}
\end{minipage}
\end{mdframed}

\FloatBarrier

In Listing \ref{lst:thompson-mem-em}, \texttt{do\_action} is a procedure which
interfaces with the outside environment and returns the reward received.
\texttt{do\_action} is wrapped in the statistical memoizer \texttt{mem\&em},
with parameters given by the semicontext \texttt{theta} (see Setion
\ref{sec:thompson-framework}).  \texttt{theta} is given a prior distribution,
and as more actions \texttt{a} are probed, inference is performed on
\texttt{theta}, and subsequent evaluations of the emulator \texttt{r\_emulate}
are affected accordingly.  The action \texttt{a} chosen at each step is
determined by the procedure \texttt{argmax}; for purposes of Thompson sampling,
\texttt{argmax} (which takes a possibly stochastic procedure as its argument)
should be implemented so that \texttt{(argmax func)} returns an approximation to
$\argmax_{\texttt{a}} \Ebkt{\texttt{(func a)}}$.

\begin{figure}[p]
  \centering
  \includegraphics[height=0.8\textheight]{figs/slide1.pdf}
  \caption{
    Top: Diagrammatic representation of Thompson sampling using the statistical
    memoizer $\mm$.  The emulator $\ftt_\emu$ is used to choose a point
    $\xtt_\rmnext$ to acquire; the acquired data are incorporated into the memo
    table and the parameters $\theta$ are updated according to this new
    information.
    Middle: Mathematical specification of the statistical emulator $\ftt_\emu$.
    Bottom: Depiction of the state of the emulator after zero, one, and two
    probes have been taken.  Here the true value function $V$ is drawn in blue;
    the believed distribution on $V$ is drawn in red.  Between probes,
    $\ftt_\emu$ can be \texttt{sample}d but not \texttt{predict}ed, as its state
    (see Section \ref{sec:gps-in-pps}) must not be changed.  The state (i.e.,
    the memo table) must only be changed when $\ftt_\probe$ is called.
  }
  \label{fig:slide1}
\end{figure}
\end{comment}
\myparagraph{A Mathematical Specification}\label{sec:math-spec}
Our mathematical specification assert the following properties:
\begin{itemize}
  \item The regression function has a Gaussian process prior.
  \item The actions $a_1,a_2,\ldots \in \Acal$ are chosen by a Metropolis-like search
    strategy with Gaussian drift proposals.
  \item The hyperparameters of the Gaussian process are inferred using
    Metropolis--Hastings sampling after each action.
\end{itemize}

In this version of Thompson sampling, the contexts $\ttheta$ are Gaussian
processes over the action space $\Acal = [-20, 20] \subseteq \R$.  That is,
\[ V \sim \GP(\mu, K), \]
where the mean $\mu$ is a computable function $\Acal \to \R$ and the covariance
$K$ is a computable (symmetric, positive-semidefinite) function $\Acal \times
\Acal \to \R$.  This represents a Gaussian process $\br{R_a}_{a \in \Acal}$,
where $R_a$ represents the reward for action $a$.  Computationally, we represent
a context as a data structure
\[ \ttheta = (\theta, \abf_\past, \rbf_\past) = (\mu_\prior, K_\prior, \eta, \abf_\past, \rbf_\past), \]
where $\mu_\prior$ is a procedure to be used as the prior mean function. w.l.o.g. we set $\mu_\prior \equiv 0$.
$K_\prior$ is a procedure to be used as the prior covariance function, parameterized by 
$\eta$.

The posterior mean and covariance for such a context $\ttheta$ are gotten by the
usual conditioning formulas (assuming, for ease of exposition as above, that the
prior mean is zero):\footnote{
  Here, for vectors $\abf = \pn{a_i}_{i=1}^{n}$ and $\abf' =
  \pn{a'_i}_{i=1}^{n'}$, $\mu(\abf)$ denotes the vector
  $\pn{\mu(a_i)}_{i=1}^{n}$ and $K(\abf,\abf')$ denotes the matrix
  $\begin{bmatrix} K(a_i, a'_j) \end{bmatrix}_{1 \leq i \leq n, 1 \leq j
  \leq n'}$.
}
\begin{align*}
  \mu(\abf)
  &= \mu\pn{\abf \mvert \abf_\past, \rbf_\past} \\
  &= K_\prior(\abf, \abf_\past)
     \,K_\prior(\abf_\past, \abf_\past)^{-1}
     \,\rbf_\past \\
  K(\abf, \abf)
  &= K\pn{\abf, \abf \mvert \abf_\past, \rbf_\past} \\
  &= K_\prior(\abf, \abf)
     - K_\prior(\abf, \abf_\past)
       \,K_\prior(\abf_\past, \abf_\past)^{-1}
       \,K_\prior(\abf_\past, \abf).
\end{align*}
Note that the context space $\Theta$ is not a finite-dimensional parametric
family, since the vectors $\abf_\past$ and $\rbf_\past$ grow as more samples are
taken.  $\Theta$ is, however, representable as a computational
procedure together with parameters and past samples, as we do in the
representation $\ttheta = (\mu_\prior, K_\prior, \eta, \abf_\past, \rbf_\past)$.

We combine the Update and Sample steps of Algorithm \ref{alg:thompson} by
running a Metropolis--Hastings (MH) sampler whose stationary distribution is the
posterior $P\pn{\theta \mvert \abf_\past, \rbf_\past}$.  The functional forms of
$\mu_\prior$ and $K_\prior$ are fixed in our case, so inference is only done
over the parameters $\eta = \br{\sigma,\ell}$; hence we equivalently write
$P\pn{\sigma,\ell \mvert \abf_\past, \rbf_\past}$ for the stationary
distribution.  We make MH proposals to one variable at a time, using the prior
as proposal distribution:
\[
  Q_\proposal\pn{\sigma',\ell \mvert \sigma,\ell} = P(\sigma')
\]
and
\[
  Q_\proposal\pn{\sigma,\ell' \mvert \sigma,\ell} = P(\ell').
\]
The MH acceptance probability for such a proposal is
\[
  P_\accept\pn{\sigma',\ell' \mvert \sigma,\ell}
  =
  \min\br{1,\ \frac{
    Q_\proposal\pn{\sigma,\ell \mvert \sigma',\ell'}
    }{
    Q_\proposal\pn{\sigma',\ell' \mvert \sigma,\ell}
    }
  \cdot
  \frac{
    P\pn{\abf_\past,\rbf_\past \mvert \sigma',\ell'}
    }{
    P\pn{\abf_\past,\rbf_\past \mvert \sigma,\ell}
    }}
\]
Because the priors on $\sigma$ and $\ell$ are uniform in our case, the term
involving $Q_\proposal$ equals $1$ and we have simply
\begin{align*}
  P_\accept\pn{\sigma',\ell' \mvert \sigma,\ell}
  &=
  \min\br{1,\ \frac{
    P\pn{\abf_\past,\rbf_\past \mvert \sigma',\ell'}
    }{
    P\pn{\abf_\past,\rbf_\past \mvert \sigma,\ell}
    }} \\[2mm]
  &=
  \min\bigg\{1,\ \exp\bigg( -\frac12\bigg(
    \rbf_\past^T K_\prior\pn{\abf_\past, \abf_\past \mvert \sigma', \ell'}^{-1} \rbf_\past \\
  & \qquad\qquad\qquad\qquad\qquad -
    \rbf_\past^T K_\prior\pn{\abf_\past, \abf_\past \mvert \sigma, \ell}^{-1} \rbf_\past
  \bigg)\bigg)\bigg\}.
\end{align*}
The proposal and acceptance/rejection process described above define a
transition operator $\tau_\update$ which is iterated a specified number of
times; the resulting state of the MH Markov chain is taken as the sampled
semicontext $\theta$ in Step \ref{itm:thompson-step-sample} of Algorithm
\ref{alg:thompson}.

For Step \ref{itm:thompson-step-search} (Search) of Thompson sampling, we
explore the action space using an MH-like transition operator $\tau_\search$.
As in MH, each iteration of $\tau_\search$ produces a proposal which is either
accepted or rejected, and the state of this Markov chain after a specified
number of steps is the new action $a$.  The Markov chain's initial state is the
most recent action, and the proposal distribution is Gaussian drift:
\[ Q_\proposal\pn{a' \mvert a} \sim \Ncal(a,\,\propstd^2), \]
where the drift width $\propstd$ is specified ahead of time.  The acceptance
probability of such a proposal is
\[ P_\accept\pn{a' \mvert a} = \min\br{1,\ \exp\pn{-E\pn{a' \mvert a}}}, \]
where the energy function $E\pn{\bullet \mvert a}$ is given by a Monte Carlo
estimate of the difference in value from the current action:
\[ E\pn{a' \mvert a} = -\frac1s \pn{\muhat(a') - \muhat(a)} \]
where
\[ \muhat(a) = \frac{1}{N_\avg} \sum_{i=1}^{N_\avg} \widetilde{r}_{i,a} \]
and
\[ \widetilde{r}_{i,a} \sim \Ncal(\mu(a), K(a,a)) \]
and $\br{\widetilde{r}_{i,a}}_{i=1}^{N_\avg}$ are i.i.d.\ for a fixed $a$.
Here the temperature parameter $s \geq 0$ and the population size $N_\avg$ are
specified ahead of time.  Proposals of estimated value higher than that of the current action are
always accepted, while proposals of estimated value lower than that of the
current action are accepted with a probability that decays exponentially
with respect to the difference in value.
The rate of the decay is determined by the temperature parameter $s$,
where high temperature corresponds to generous acceptance probabilities.
For $s=0$, all proposals of lower value are rejected; for $s=\infty$, all
proposals are accepted.
For points $a$ at which the posterior mean $\mu(a)$ is low but the
posterior variance $K(a,a)$ is high, it is possible (especially when
$N_\avg$ is small) to draw a ``wild'' value of $\muhat(a)$, resulting in a
favorable acceptance probability.




 \begin{figure}
  \setlength{\tabcolsep}{1pt} 
 \centering
\begin{tabular}{rl}
\includegraphics[width=0.6\textwidth]{figs/bayesopt/iteration_vs_error.png}&\includegraphics[width=0.4\textwidth]{figs/bayesopt/legend.png}
\put(-126,133){\scriptsize Trimodal ground truth, drift proposal}
\put(-126,125){\scriptsize Trimodal ground truth, uniform proposal}
\put(-126,117){\scriptsize Bimodal ground truth, uniform proposal}
\put(-126,108){\scriptsize Bimodal ground truth, drift proposal}
\put(-126,100){\scriptsize Local Optima, trimodal function}
\put(-126,92){\scriptsize Local Optima, bimodal function}
\put(-119,78){\scriptsize Ground Truth}
\put(-119,66){\scriptsize Posterior samples}
\put(-119,55){\scriptsize Next Probe}
\put(-119,43){\scriptsize Estimated Optimum}
\put(-119,31){\scriptsize Past Probes}
\end{tabular}
\begin{tabular}{llll}\hline
\multicolumn{1}{|l|}{\raisebox{-0.5\height}{\includegraphics[width=0.3\textwidth]{figs/bayesopt/opt_sequence_uniform_trimodal_7.png}}}&
\multicolumn{1}{l|}{\raisebox{-0.5\height}{\includegraphics[width=0.3\textwidth]{figs/bayesopt/opt_sequence_uniform_trimodal_8.png}}}&
\multicolumn{1}{l|}{\raisebox{-0.5\height}{\includegraphics[width=0.3\textwidth]{figs/bayesopt/opt_sequence_uniform_trimodal_9.png}}} \\ \hline
\multicolumn{1}{|l|}{\raisebox{-0.5\height}{\includegraphics[width=0.3\textwidth]{figs/bayesopt/opt_sequence_uniform_trimodal_10.png}}}&
\multicolumn{1}{l|}{\raisebox{-0.5\height}{\includegraphics[width=0.3\textwidth]{figs/bayesopt/opt_sequence_uniform_trimodal_11.png}}}&
\multicolumn{1}{l|}{\raisebox{-0.5\height}{\includegraphics[width=0.3\textwidth]{figs/bayesopt/opt_sequence_uniform_trimodal_12.png}}} \\ \hline
\end{tabular}
\put(-360,56){\footnotesize$i = 7$}
\put(-238,56){\footnotesize$i = 8$}
\put(-117,56){\footnotesize$i = 9$}
\put(-360,-8){\footnotesize$i = 10$}
\put(-238,-8){\footnotesize$i = 11$}
\put(-117,-8){\footnotesize$i = 12$}
\caption{Top: the estimated optimum over time. Blue and Red represent optimization with uniform and Gaussian drift proposals. Black lines indicate the local optima of the true functions. Bottom: a sequence of actions. Depicted are iterations 7-12 with uniform proposals.}
\end{figure}
Indeed, taking an action $a$ with low estimated value but high uncertainty
serves the useful function of improving the accuracy of the estimated value
function at points near $a$ (see Figure \ref{fig:slide2}).\footnote{
  At least, this is true when we use a smoothing prior covariance function such
  as the squared exponential.
}$^,$\footnote{
  For this reason, we consider the sensitivity of $\muhat$ to uncertainty to be
  a desirable property; indeed, this is why we use $\muhat$ rather than the
  exact posterior mean $\mu$.
}
We see a comlete probabilistic program with \gpmem\ implementing Bayesian optimization
with Thompson Sampling below (Listing \ref{alg:bayesopt}).
 \begin{mdframed}
\begin{minipage}{\linewidth}
\small
\belowcaptionskip=-10pt
\begin{lstlisting}[caption={Bayesian optimization using \gpmem},mathescape,numbers=none,label=alg:bayesopt,escapechar=\#]
#\linenumber{1}#assume sf = tag("hyper", 0, uniform_continuous(0, 10))
#\linenumber{2}#assume l = tag("hyper", 1, uniform_continuous(0, 10))
#\linenumber{3}#assume se = make_squaredexp(sf, l)
#\linenumber{4}#assume blackbox_f = get_bayesopt_blackbox()
#\linenumber{5}#assume (f_compute, f_emulate) = gpmem(blackbox_f, se)

// A naive estimate of the argmax of the given function
#\linenumber{6}#define mc_argmax = proc(func) {
#\linenumber{7}#  candidate_xs = mapv(proc(i) {uniform_continuous(-20, 20)},
#\linenumber{8}#                      arange(20));
#\linenumber{9}#  candidate_ys = mapv(func, candidate_xs);
#\linenumber{10}#  lookup(candidate_xs, argmax_of_array(candidate_ys))
#\linenumber{11}#};

// Shortcut to sample the emulator at a single point without packing
// and unpacking arrays
#\linenumber{12}#define emulate_pointwise = proc(x) {
#\linenumber{13}#  run(sample(lookup(f_emulate(array(unquote(x))), 0)))
#\linenumber{14}#};

// Main inference loop
#\linenumber{15}#infer repeat(15, do(pass,
  // Probe V at the point mc_argmax(emulate_pointwise)
#\linenumber{16}#  predict(f_compute(unquote(mc_argmax(emulate_pointwise)))),
  // Infer hyperparameters
#\linenumber{17}#  mh("hyper", one, 50)));
\end{lstlisting}

\end{minipage}
\end{mdframed}

\section{Discussion}
This paper has shown that it is feasible and useful to embed Gaussian processes
in higher-order probabilistic programming languages by treating them as a kind
of statistical memoizer. It has described classic GP regression with both fully
Bayesian and MAP inference in a hierarchical hyperprior, as well as state-of-the-art
applications to discovering symbolic structure in time series and to Bayesian optimization.
All the applications share a common 100-line Python GP library and require fewer than 20 lines
of probabilistic code each.

These results suggest several research directions. First, it will be important
to develop versions of {\tt gpmem} that are optimized for larger-scale
applications. Possible approaches include the standard low-rank approximations
to the kernel matrix that are popular in machine learning~\citep{bui2014tree} as well
as more sophisticated sampling algorithms for approximate conditioning of the
GP~\citep{lawrence2009efficient}.
Second, it seems fruitful to abstract the notion of a ``generalizing" memoizer
from the specific choice of a Gaussian process model as the mechanism for
generalization. ``Generalizing" or statistical memoizers with custom regression techniques could be broadly useful in performance engineering and scheduling systems.
The timing data from performance benchmarks could be run through a generalizing memoizer by default.
This memoizer could be queried (and its output error bars examined) to inform the best strategy
for performing the computation or predict the likely runtime of long-running jobs.
 Third, the structure learning application suggests follow-on research in information
retrieval for structured data. It should be possible to build a time series search engine
that can handle search predicates such as ``has a rising trend starting around
1988" or ``is perodic during the 1990s".
The variation on the Automated Statistician presented in this paper can provide ranked result
sets for these sorts of queries because it tracks posterior uncertainty over structure and also
because the space of structural patterns that it can handle is easy to modify by making small
changes to a short VentureScript program.


The field of Bayesian nonparametrics offers a principled, fully Bayesian
response to the empirical modeling philosophy in machine learning~\citep{ghahramani2013bayesian},
where Bayesian inference is used to encode a state of broad ignorance rather
than a bias stemming from strong prior knowledge. It is perhaps surprising that
two key objects from Bayesian nonparametrics, Dirichlet processes and \ac{GP}s,
fit naturally in probabilistic programming as variants of
memoization~\citep{roy2008stochastic}. It is not yet clear if the same will be true
for other processes, e.g. Wishart processes, or hierarchical Beta processes. We hope that the results in this paper encourage the development of other nonparametric libraries for higher-order probabilistic programming languages.

\myparagraph{Acknowledgements}
This research was supported by DARPA
  (under the XDATA and PPAML programs), IARPA (under research contract
  2015-15061000003), the Office of Naval Research (under research
  contract N000141310333), the Army Research Office (under agreement
  number W911NF-13-1-0212), the Bill \& Melinda Gates Foundation, and
  gifts from Analog Devices and Google.
\newpage
\section*{Appendix}
\subsection*{A Covariance Functions}
SE and WN are defined in the text above, for completeness we will introduce the covariance:
% ToDo: check if this should be removed
\begin{equation}
k_{LIN}(x,x^\prime) = \theta (x x^\prime) 
\end{equation}
\begin{equation}
k_{PER}(x,x^\prime) = \theta \exp \bigg( \frac{2 \sin^2 ( \pi (x - x^\prime)/p}{\ell^2} \bigg) 
\end{equation}
\begin{equation}
k_{RQ}(x,x^\prime) =   \theta \bigg(1 + \frac{(x - x^\prime)^2}{2 \alpha \ell^2} \bigg)^{-\alpha}
\end{equation}


\subsection*{B Covariance Simplification}
\begin{minipage}{\linewidth}
\small
\belowcaptionskip=-10pt
\begin{lstlisting}[frame=single,mathescape,label=alg:simplify,basicstyle=\selectfont\ttfamily]

SE $\times$ SE                  $\rightarrow$ SE 
{SE,PER,C,WN} $\times$ WN       $\rightarrow$ WN
LIN $+$ LIN                $\rightarrow$ LIN
{SE,PER,C,WN,LIN} $\times$ C    $\rightarrow$  {SE,PER,C,WN,LIN} 
\end{lstlisting}
\end{minipage}
Rule 1 is derived as follows:
\begin{equation}
\begin{aligned}
\sigma_c^2 \exp(-\frac{(x-x^\prime)^2}{2\ell_c^2})  &=  \sigma_a^2 \exp(-\frac{(x-x^\prime)^2}{2\ell_a^2}) \times  \sigma_b^2 \exp(-\frac{(x-x^\prime)^2}{2\ell_b^2}) \\
&= \sigma_c^2 \exp(-\frac{(x-x^\prime)^2}{2\ell_a^2}) \times   \exp(-\frac{(x-x^\prime)^2}{2\ell_b^2}) \\
&= \sigma_c^2 \exp \bigg(-\frac{(x-x^\prime)^2}{2\ell_a^2} -\frac{(x-x^\prime)^2}{2\ell_b^2}\bigg) \\
&= \sigma_c^2 \exp \bigg(-\frac{(x-x^\prime)^2}{2\ell_c^2}\bigg) \\
\end{aligned}
\end{equation}
Rule 3 is derived as follows:
\begin{equation}
 \theta_c (x \times x^\prime) = \theta_a (x \times x^\prime) + \theta_b (x \times x^\prime) 
\end{equation}
For stationary kernels that only depend on the lag vector between $x$ and $x^\prime$ it holds that multiplying such a kernel with a WN kernel we get another WN kernel. Take for example the SE kernel:
\begin{equation}
 \sigma_a^2 \exp \bigg(-\frac{(x-x^\prime)^2}{2\ell_c^2}\bigg) \times  \sigma_b \delta_{x,x^\prime} =  \sigma_a \sigma_b \delta_{x,x^\prime}
\end{equation}
Multiplying any kernel with a constant obviously changes only the scale parameter of a kernel.


\subsection*{C The Struct-Operator}

\begin{align*}
\Struct(\klin) &= \text{LIN}\\
\Struct(\kper) &= \text{PER}\\
\Struct(\kse) &= \text{SE}\\
\Struct(\kwn) &= \text{WN}\\
\Struct(k^{\text{linear+periodic}}) &= \text{LIN}+\text{PER}\\
\Struct(k^{\text{linear} \times \text{periodic}}) &= \text{LIN}
\times\text{PER}
\end{align*}




\newpage
\subsection*{D Glossary}
\begin{tabular}{l l}
$\mathcal{N}$		&  (Multivariate-) Gaussian \\
$\mathcal{GP}$		&  Gaussian Process \\
$\mathbb{E}$		&  Expectation    \\
$x,x_i$ 			&  Scalar, possibly indexed with $i$   \\
$\xbf$				&  Column vector, training data:
regression input (also actions in section \ref{sec:thompson})    \\
$\ybf$				&  Column vector, training data:
regression output  (also rewards in section \ref{sec:thompson})    \\
$\Acal$				& A set of possible actions \\

 $\xprime$			&  Column vector, unseen test input: regression input    \\
 $\yprime$			&  Column vector, sample from predictive posterior, that
is a sample from\\
&$\mathcal{N}(\mupost,\Kpost)$    \\

$\xstar$         &  Column vector, unseen test
input: regression input before any data\\
&has been observed    \\
$\ystar$		&  Column vector, sample from the predictive prior
conditioned on $\thetabf$\\
& and unseen test input $\xstar$ \\
$\Dbf$		&  Data matrix $[\xbf\; \ybf]$ \\

$\mu(x)$                 &  Mean function \\
$\thetabf_{\text{mean}}$	&  hyper-parameters for a mean function
\\
$k \text{ or } k(x_i,x_j)$	&  a covariance function or kernel, that is a
function that takes two scalars as input \\

$\thetabf$			&  hyper-parameters for a
kernel/covariacne function (also semicontext in section \ref{sec:thompson})\\
$k(x_i,x_j \midtheta )$		&  a kernel conditioned on its
hyper-parameters \\
$K(\xbf,\xprime \midtheta )$    &  Function outputting a matrix of dimension $I \times J$
with entries $k(x_i,x_j \midtheta )$;\\
&with $x_i \in \xbf$ and $x_j \in \xprime$ where $I$ and $J$ indicate the length
of the\\
&column vectors $\xbf$ and
$\xprime$ \\

$\ktheta$			&  a covariance function parameterized with
$\thetabf$\\

$\Ktheta\text{ or }\Kbf_{(\thetabf,\xbf,\xbf)}$		&  covariance matrix
computed with by $K(\xbf,\xbf \midtheta)$   \\
$\mupost$ &  Posterior mean vector for $\yprime \mid \xbf, \xprime, \ybf, \thetabf$ \\
$\Kpost$     &  Posterior covariance matrix for $\yprime \mid \xbf, \xprime, \ybf, \thetabf$ \\
$\Lbf$			&lower triangular matrix, given by the Cholesky factorization as 
$\Lbf\coloneqq \text{chol}(\Ktheta)$\\

$\kse$ &  Squared exponential covariance function  \\
$\klin$ &  Linear covariance function  \\
$k^{\text{constant}}$ &  Constant covariance function  \\
$\kwn$ &  White noise covariance function  \\
$k^{\text{rational quadratic}}$  &  Rational quadratic covariance function  \\
$\kper$ &  Periodic covariance function  \\

SE			&  Symbolic expression for the squared exponential covariance function  \\
LIN		        &  Symbolic expression for the linear covariance function  \\
PER		        &  Symbolic expression for the periodic covariance function  \\
RQ		        &  Symbolic expression for the rational quadratic covariance function  \\
C		        &  Symbolic expression for the constant covariance function  \\
WN		        &  Symbolic expression for the white noise covariance function  \\
\end{tabular}
\newpage
\begin{tabular}{l l}
$\land$                &  Logical and \\
$\lor$                 &  Logical or \\
$\Krv$			& Random variable over kernel functions \\
kernel functions \\
Parse$(k)$		& Parse the structure for kernel $k$ \\
Simplify$(k)$		& Simplify the functional expression for kernel $k$ \\
Struct$(k)$		& Symbolic interpretation for kernel $k$ \\
$\Cont(k,k^t)$          & A kernel $k^t$ contains the global kernel structure
$k$\\
$\SEK$			&  Operator to check if $\Struct(\Ksrv=k)=\text{SE}$\\
$\LINK$			&  Operator to check if $\Struct(\Ksrv=k)=\text{LIN}$\\
$\PERK$			&  Operator to check if $\Struct(\Ksrv=k)=\text{PER}$\\
$\WNK$			&  Operator to check if $\Struct(\Ksrv=k)=\text{WN}$\\
BK			& A set of base kernels \\
$\Sbf$			& A subset of BK, randomly selected \\
$\Omegabf$		& Random variable for composition operators, in our
case that is kernel addition\\
&and multiplication $\{+,\times\}$\\
$\Gamma(\alpha,\beta)$ & Gamma distribution with shape parameter $\alpha$ and
rate $\beta$ \\

$\ell$			& Length-scale parameter for $\kse$  \\
$sf$			& Scale factor parameter\\
$\ttheta \in \Theta$    & Context in Thompson sampling\\ 
$\Theta$    & Context space\\ 
$V$			& Value function\\
$Q_\proposal$		& Proposal distribution\\
$E$			& Energy function\\
$s$			& Temperature parameter \\
\end{tabular}

%\subsection*{C Additional Structure Learning Results}
%Below we depict additional results for the airline data set using a hypothesis space of LIN, PER, SE and RQ covariance functions as base kernels for the composition. Fig \ref{fig:airline1} shows the training data and posterior samples drawn from the \ac{GP} with posterior composite structure. 

\begin{figure}
\includegraphics[width=\textwidth]{figs/posterior_samples_airline.png}
\caption{Data (black x) and posterior samples (red) for the airline data set.}\label{fig:airline1}
\end{figure}
Fig. \ref{fig:posterior_twosamples} shows why it advantageous to be Bayesian here. We see two possible hypotheses for the composite structure of $\mathbf{K}$. 

\begin{figure}
        \centering
        \begin{subfigure}{0.49\textwidth} \centering
                \includegraphics[width=0.9\textwidth]{figs/airline_struct_1.pdf}
        \end{subfigure}
	\begin{subfigure}{0.49\textwidth} \centering
                \includegraphics[width=0.8\textwidth]{figs/airline_struct_2.pdf}
        \end{subfigure}
        \caption{We see two possible hypotheses for the composite structure of $\mathbf{K}$. (a) Most frequent sample drawn from the posterior on structure. We have found two global components. First, a smooth trend (LIN $\times$ SE) with a non-linear increasing slope. Second, a periodic component with increasing variation and noise. (b) Second most frequent sample drawn from the posterior on structure. We found one global component. It is comprised of local changes that are periodic and with changing variation.}\label{fig:posterior_twosamples}
\end{figure}

\newpage
\bibliography{VentureGP.01}
\bibliographystyle{apalike}
\end{document}
